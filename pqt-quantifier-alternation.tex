

\subsection*{Quantifier alternation}
Consider the following statements involving alternation of quantifiers.
\begin{itemize}
\item
Everyone loves someone (or other).
\[S_1:\ \ \ (\forall x)(\exists y)(x\ \mathrm{loves}\ y).\]
\item
There is someone whom everyone loves.
\[S_2:\ \ \ (\exists y)(\forall x)(x\ \mathrm{loves}\ y).\]
\item
Everyone is loved by someone.
\[S_3:\ \ \ (\forall y)(\exists x)(x\ \mathrm{loves}\ y).\]
\item
Someone loves everyone.
\[S_4:\ \ \ (\exists x)(\forall y)(x\ \mathrm{loves}\ y).\]
\end{itemize}
Of course, each of these statements means something different (ie none are equivalent) - and so we see immediately that \emph{the order in which quantifiers appear matters}. 

\begin{aside}
    Give an intuitive argument to show, although no two of the above statements are equivalent, that $S_2$ implies $S_1$, and that $S_4$ implies $S_3$. 
\end{aside}

Aside from the implications $S_2 \implies S_1$ and $S_4 \implies S_3$, no other implications hold. To see this, we give structures which act as a counterexamples to each purported implication. Consider the following three structures $A, B, C$. 
\[
\begin{array}{| c | c | c |}
\hline
 \mbox{Structure}& \mbox{Universe} & \mbox{Extension of }L\\
  \hline            
  A & \{a,b\} & \{\op{a}{a},\op{b}{b}\}\\
  \hline
  B & \{a,b\} & \{\op{b}{b},\op{a}{b}\}\\
 \hline
 C & \{a,b\} & \{\op{b}{b},\op{b}{a}\}\\
 \hline  
\end{array}
\]
Note that $A\models S_1$ and $A\models S_3$, while $A\not\models S_2$ and $A\not\models S_4$, from which it follows, by definition, that $S_1$ does not imply $S_2$, nor does $S_3$ imply $S_4$. Moreover $B\models S_2$, but $B\not\models S_3$, and $C\models S_4$, but $C\not\models S_1$; thus $S_2$ does not imply $S_3$, and $S_4$ does not imply $S_1$. 

Failure of the remaining (non-trivial) implications now follows. For example, $S_1$ does not imply $S_4$. To see this, suppse \emph{ad reductio} that $S_1$ implies $S_4$. Then since $S_2$ implies $S_1$ and $S_4$ implies $S_3$, it follows by the transitivity of implication that $S_2$ implies $S_3$. But we have already seen that $S_2$ does not imply $S_3$ ($B$ was the counterexample), a contradiction.

\begin{aside}
    Verify that each of the statements above is true. For example, if we claimed that $A \models S_i$ for some $i$, explain in your own words why, in fact, $A$ actually models $S_i$.  

    Show that the remaining non-trivial implications also fail. To do this, use proof-by-contradiction as we did to show that $S_1$ does not imply $S_4$. 
\end{aside}

We summarize the results of this discussion in the following matrix $\langle a_{ij}\mid 1\leq i,j\leq 4\rangle$, where $a_{ij} = 1$ if and only if the schema in the $i$-th row implies the schema in the $j$-th column.
\[
\begin{array}{| c | c | c | c | c|}
\hline
 S_i\mbox{ implies }S_j& S_1 & S_2 & S_3 & S_4\\
  \hline            
  S_1 & 1 & 0 & 0 & 0\\
  \hline
  S_2  & 1 & 1 & 0 & 0\\
 \hline
 S_3 & 0 & 0 & 1 & 0\\
 \hline
 S_4 & 0 & 0 & 1 & 1\\
 \hline  
\end{array}
\]


\subsection*{Scope ambiguity}
As a quick aside, we'll now explore ``scope ambiguities.'' Consider the statement, ``everybody loves a lover.'' Let's use the binary redicate $L$ to denote love, ie ``$x$ is a lover'' can be schematized as $(\exists y)Lxy$\footnote{There are two main styles for how to write an element in a binary relation. Here, we are using $Lxy$ to indicate that the relation $L$ holds between $x$ and $y$; this could also be written as $xLy$. We choose the former. When the relation is more naturally written in the center we will do so - ie for the relation ``less than'', we will write $x < y$ rather than $<xy$}. 


There are two possible\footnote{Well, there are infinitely many \emph{possible} readings of that sentence. There are only two \emph{reasonable} ones} readings of the sentence ``everybody loves a lover'': either we interpret it as ``everybody loves someone who is a lover'', and ``if someone is a lover, then everybody loves her''. Corresponding to these two interpretations, we have the respective schematizations:

\[(\forall z)(\exists x)((\exists y)Lxy\wedge Lzx)\footnote{A decent reading of this schematization is ``for every person $z$, there is a person $x$ such that $z$ loves $x$ and there is a $y$ that $x$ loves''.}\] 
and 
\[(\forall x)((\exists y)Lxy\supset (\forall z)Lzx)\]


We claim that a structure $A$ satisfies the second schema if and only if either $L^A$ is empty or $L^A=U^A\times U^A$, the cartesian product of the universe of $A$ with itself. 

\begin{aside}
    Give an intuitive argument to verify this claim. 
\end{aside}

On the other hand, if a structure $B$ satisfies the first schema, then $L^B$ is non-empty; moreover, if $B$ consists of a pair of requiting lovers at least one of whom is not a narcissist, $B$ satisfies the first, but not the second, schema. Thus, neither disambiguation of the original sentence implies the other.

\begin{aside}
    Give two different schematizations of the sentence ``all that glitters is not gold'', corresponding to two different interpretations of the scope of the negation ``not''. 
\end{aside}