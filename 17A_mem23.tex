\iffalse
Now that we have concluded our analysis of definability over $B$ with the help of the Compactness Theorem, let's begin our exploration of a formal system of deduction for polyadic quantification theory. We'll start by considering the notion of a formal system in general terms, and articulate an abstract formulation of the goal of the project.

The elaboration of formal systems of deduction for polyadic quantification caps a long effort to achieve the highest possible degree of rigor in mathematical argumentation. This search was in part motivated by the periodic appearance of contradictions in the mathematical theory of the continuum. This theory, whose genesis may be dated to the Pythagoreans' proof that the square root of two is irrational, was developed with great vigor in the seventeenth century, in connection with the rise of the new physics and its effort to provide a unified theory of the motion of both terrestrial and celestial bodies. As mathematical analysis (as the theory of the continuum came to be called) developed in the nineteenth century, and became ever more enmeshed with new areas of physics, such as the theory of heat, the need for a more rigorous foundation for the subject became ever more pressing. In particular, even the greatest of mathematicians, such as Augustin Cauchy, were hampered by the lack of a perspicuous notation for iterated quantification in formulating suitable convergence conditions guaranteeing continuity for the limits of sequences of functions. Throughout the nineteenth century several mathematicians, among them Bernard Bolzano, Georg Cantor, Cauchy, and Richard Dedekind, strove to place the subject of analysis on a firm footing by reducing the the theory of the continuum to the theory of the integers (arithmetic) through the use of sets or sequences of rational numbers; the outcome of these efforts came to be known as ``the arithmetization of analysis'' and was celebrated by David Hilbert, in his famous 1900 address to the International Congress of Mathematicians held in Paris, as one of the great achievements of nineteenth-century mathematics. Late in the century, Gottlob Frege sought an even greater economy in the basic principles required for the rigorous foundation of analysis through his attempt to reduce arithmetic to logic. Though this effort was ultimately doomed by Russell's paradox, Frege's articulation of a calculus for logical deduction was a signal achievement in the development of modern logic. In reaction to the paradoxes, Hilbert, in collaboration with various of his students, and a number of other mathematicians, developed \emph{formal systems} of logic of the sort expounded in contemporary treatments deductive logic such as Goldfarb's text. 

From an epistemological point of view, one might insist that a mathematical proof should be self-certifying, that is, if the derivation \der{d}\ is a proof of the mathematical statement \stat{s}, then this should be immediately recognizable -- no further argument should be required to convince someone of this, for otherwise, it is not \der{d}\ itself, but only \der{d}\ supplemented with this additional argument, that constitutes a proof of \stat{s}. The notion of formal system takes this insistence to a natural limit: in a formal system the relation ``\der{d}\ is a proof of \stat{s}'' is mechanically decidable, that is, there is an algorithm which can be applied to the pair $\op{\der{d}}{\stat{s}}$ to determine whether the proof relation obtains. In a formal system of deduction $\forms{F}$ a derivation \der{d} consists of a finite sequence of schemata, and a statement \stat{s} is represented by a schema $S$. We write $\Pi_{\forms{F}}(\der{d},S)$ for ``\der{d}\ is a proof of $S$ in the formal system \forms{F}.'' The schema $S$ is a theorem of \forms{F}\ if and only if there is a derivation \der{d}\ such that $\Pi_{\forms{F}}(\der{d},S)$. We write $\vdash_{\forms{F}}S$ for ``S is a theorem of \forms{F}.'' In like fashion, we write $X\vdash_{\forms{F}}S$ for ``S is derivable from hypotheses $X$ in \forms{F}.''

\begin{center}
``Everybody loves a lover; I'm a lover'' implies \\
``everybody loves everyone.'' \\
$\{(\forall x) ((\exists y) Lxy \supset (\forall z) Lzx), (\exists x)(\exists
y) Lxy \}$ implies $(\forall v)(\forall z) Lvz.$
\end{center}
\[
\begin{array}{lll}
\{1\}   & (1)\  (\exists x)(\exists y) Lxy &  \mathrm{P}\\
\{1,2\}   & (2)\ (\exists y) Lwy  & (1)w\ \mathrm{EII}\\
\{3\}   & (3)\ (\forall x) ((\exists y) Lxy \supset   & 
\mathrm{P}\\
  &\ \ \ \  (\forall z) Lzx)  & \\
\{3\}   & (4)\ (\exists y) Lwy \supset   & (3)\ \mathrm{UI}\\
  &\ \ \ \ (\forall z) Lzw & \\
\{1,2,3\}   & (5)\ (\forall z) Lzw  & (2)(4)\ \mathrm{TF}\\
\{1,2,3\}   & (6)\ Lvw  & (5)\ \mathrm{UI}\\
\{1,\not 2,3\}   & (7)\ (\exists y) Lvy  & (5)\ \mathrm{EG};\{2\}\
\mathrm{EIE}\\ 
\{3\}   & (8)\ (\exists y) Lvy \supset (\forall z) Lzv  & (3)\ \mathrm{UI}\\
\{1,3\}   & (9)\  (\forall z) Lzv & (7)(8)\ \mathrm{TF}\\
\{1,3\}   & (10)\  (\forall v)(\forall z) Lzv & (9)\ \mathrm{UG}
\end{array}
\]

We commenced our study of the deductive apparatus for polyadic quantification theory expounded in Warren Golfarb's text \emph{Deductive Logic}. We gave examples of deductions using the rules described on pages 183 -- 185 of the text.
We began with a simple to deduction to show that if a relation is asymmetric, then it is irreflexive.
\begin{center}
$\{(\forall x)(\forall y)(Lxy\supset\neg Lyx)\}$ implies $(\forall x)\neg Lxx.$
\end{center}
\[
\begin{array}{lll}
\{1\}   & (1)\  (\forall x)(\forall y)(Lxy\supset\neg Lyx) &  \mathrm{P}\\
\{1\}   & (2)\ (\forall y)(Lxy\supset\neg Lyx) & (1) \ \mathrm{UI}\\
\{1\}   & (3)\ Lxx\supset\neg Lxx &  (2)\ \mathrm{UI}\\
\{1\}   & (4)\ \neg Lxx   & (3)\ \mathrm{TF}\\
\{1\}   & (5)\ (\forall x) \neg Lxx  & (4)\ \mathrm{UG}
\end{array}
\]

We next showed that if a relation is transitive and irreflexive, then it's asymmetric.
\begin{center}
$\{(\forall x)(\forall y)(\forall z)(Lxy\supset(Lyz\supset Lxz)), (\forall x)\neg Lxx\}$ implies $(\forall x)(\forall y)(Lxy\supset\neg Lyx).$
\end{center}
\[
\begin{array}{lll}
\{1\}   & (1)\  (\forall x)(\forall y)(\forall z)(Lxy\supset(Lyz\supset Lxz)) &  \mathrm{P}\\
\{1\}   & (2)\ (\forall y)(\forall z)(Lxy\supset(Lyz\supset Lxz)) & (1) \ \mathrm{UI}\\
\{1\}   & (3)\ (\forall z)(Lxy\supset(Lyz\supset Lxz)) &  (2)\ \mathrm{UI}\\
\{1\}   & (4)\ Lxy\supset(Lyx\supset Lxx)   & (3)\ \mathrm{UI}\\
\{5\}   & (5)\ (\forall x) \neg Lxx  & \ \mathrm{P}\\
\{5\}   & (6)\ \neg Lxx  & (5)\ \mathrm{UI}\\
\{1,5\}   & (7)\ (Lxy\supset\neg Lyx)  & (4,6)\ \mathrm{TF}\\
\{1,5\}   & (8)\ (\forall y)(Lxy\supset\neg Lyx)  & (7)\ \mathrm{UG}\\
\{1,5\}   & (9)\ (\forall x)(\forall y)(Lxy\supset\neg Lyx)  & (8)\ \mathrm{UG}
\end{array}
\]
\fi

We continued our study of deduction, and began by showing that if a relation is transitive and irreflexive, then it's asymmetric.
\begin{center}
$\{(\forall x)(\forall y)(\forall z)(Lxy\supset(Lyz\supset Lxz)), (\forall x)\neg Lxx\}$ implies $(\forall x)(\forall y)(Lxy\supset\neg Lyx).$
\end{center}
\[
\begin{array}{lll}
\{1\}   & (1)\  (\forall x)(\forall y)(\forall z)(Lxy\supset(Lyz\supset Lxz)) &  \mathrm{P}\\
\{1\}   & (2)\ (\forall y)(\forall z)(Lxy\supset(Lyz\supset Lxz)) & (1) \ \mathrm{UI}\\
\{1\}   & (3)\ (\forall z)(Lxy\supset(Lyz\supset Lxz)) &  (2)\ \mathrm{UI}\\
\{1\}   & (4)\ Lxy\supset(Lyx\supset Lxx)   & (3)\ \mathrm{UI}\\
\{5\}   & (5)\ (\forall x) \neg Lxx  & \ \mathrm{P}\\
\{5\}   & (6)\ \neg Lxx  & (5)\ \mathrm{UI}\\
\{1,5\}   & (7)\ (Lxy\supset\neg Lyx)  & (4,6)\ \mathrm{TF}\\
\{1,5\}   & (8)\ (\forall y)(Lxy\supset\neg Lyx)  & (7)\ \mathrm{UG}\\
\{1,5\}   & (9)\ (\forall x)(\forall y)(Lxy\supset\neg Lyx)  & (8)\ \mathrm{UG}
\end{array}
\]

We next gave a classic ``argument by cases.''
\begin{center}
$\{(\forall x)Fx\vee(\forall x)Gx\}$ implies $(\forall x)(Fx\vee Gx).$
\end{center}
\[
\begin{array}{lll}
\{1\}   & (1)\  (\forall x)Fx\vee(\forall x)Gx &  \mathrm{P}\\
\{2\}   & (2)\ (\forall x)Fx &  \ \mathrm{P}\\
\{2\}   & (3)\ Fx &  (2)\ \mathrm{UI}\\
\{2\}   & (4)\ Fx\vee Gx   & (3)\ \mathrm{TF}\\
\{2\}   & (5)\ (\forall x) (Fx\vee Gx)  & (4)\ \mathrm{UG}\\
\{\}   & (6)\ (\forall x)Fx\supset(\forall x) (Fx\vee Gx)   & \{2\}(5)\ \mathrm{D}\\
\{7\}   & (7)\ (\forall x)Gx &  \ \mathrm{P}\\
\{7\}   & (8)\ Gx &  (7)\ \mathrm{UI}\\
\{7\}   & (9)\ Fx\vee Gx   & (8)\ \mathrm{TF}\\
\{7\}   & (10)\ (\forall x) (Fx\vee Gx)  & (9)\ \mathrm{UG}\\
\{\}   & (11)\ (\forall x)Gx\supset(\forall x) (Fx\vee Gx)   & \{7\}(10)\ \mathrm{D}\\
\{1\}   & (12)\ (\forall x) (Fx\vee Gx)  & (1,6,11)\ \mathrm{TF}\\
\end{array}
\]

We followed with an example (to be concluded next time) of argument by \emph{reductio ad absurdum}, that, in addition, illustrated use of the ``conversion of quantifiers'' rule, also known as ``driving a negation across a quantifier.'' 
\begin{center}
$(\exists y)(Py \supset (\forall x)Px)$ is valid
\end{center}
\[
\begin{array}{lll}
\{1\}   & (1)\ \neg (\exists y)(Py \supset (\forall x)Px)  & \mathrm{P}\\
\{1\}   & (2)\ (\forall y) \neg (Py \supset (\forall x)Px)  & (1)\
\mathrm{CQ}\\ 
\{1\}   & (3)\ \neg (Py \supset (\forall x)Px)  & (2)\ \mathrm{UI}\\
\{1\}   & (4)\ Py  & (3)\ \mathrm{TF}\\
\{1\}   & (5)\ (\forall x)Px  & (4)\ \mathrm{UG}\\
\{1\}   & (6)\ \neg (\forall x)Px \wedge (\forall x)Px  & (3)(5)\ \mathrm{TF}\\
\{\}   & (7)\ \neg (\exists y)(Py \supset (\forall x)Px) \supset & \{1\}(6)\
\mathrm{D}\\ 
  &\ \ \ \ (\neg (\forall x)Px \wedge (\forall x)Px) \\
\{\}   & (8)\ (\exists y)(Py \supset (\forall x)Px)  & (7)\ \mathrm{TF}
\end{array}
\]

Here are a pair of deductions that legitimate extending the ``conversion of quantifiers rule'' to allow passing directly from $\neg(\forall x)S$ to $(\exists x)\neg S$ and \emph{vice versa}. They provide further illustration of argument by \emph{reductio ad absurdum}. I include them here, though we will not cover them in class.
\begin{center}
$\{\neg(\forall x)Fx\}$ implies $(\exists x)\neg Fx.$
\end{center}
\[
\begin{array}{lll}
\{1\}   & (1)\  \neg(\forall x)Fx &  \mathrm{P}\\
\{2\}   & (2)\ \neg(\exists x)\neg Fx & (1) \ \mathrm{P}\\
\{2\}   & (3)\ (\forall x)\neg\neg Fx &  (2)\ \mathrm{CQ}\\
\{2\}   & (4)\ \neg\neg Fx   & (3)\ \mathrm{UI}\\
\{2\}   & (5)\ Fx   & (4)\ \mathrm{TF}\\
\{2\}   & (6)\ (\forall x)Fx  & (5)\ \mathrm{UG}\\
\{1,2\}   & (7)\ (\forall x)Fx\wedge\neg(\forall x)Fx  & (6)\ \mathrm{TF}\\
\{1\}   & (8)\ \neg(\exists x)\neg Fx\supset((\forall x)Fx\wedge\neg(\forall x)Fx)  & \{2\}(7)\ \mathrm{D}\\
\{1\}   & (9)\ (\exists x)\neg Fx  & (8)\ \mathrm{TF}
\end{array}
\]
\begin{center}
$\{(\exists x)\neg Fx\}$ implies $\neg(\forall x)Fx.$
\end{center}
\[
\begin{array}{lll}
\{1\}   & (1)\  (\forall x)Fx &  \mathrm{P}\\
\{2\}   & (2)\ (\exists x)\neg Fx & (1) \ \mathrm{P}\\
\{1\}   & (3)\ Fx &  (1)\ \mathrm{UI}\\
\{1\}   & (4)\ \neg\neg Fx   & (3)\ \mathrm{TF}\\
\{1\}   & (5)\ (\forall x)\neg\neg Fx   & (4)\ \mathrm{UG}\\
\{1\}   & (6)\ \neg(\exists x)\neg Fx  & (5)\ \mathrm{CQ}\\
\{1,2\}   & (7)\ \neg(\exists x)\neg Fx\wedge(\exists x)\neg Fx  & (6)\ \mathrm{TF}\\
\{2\}   & (8)\ (\forall x)Fx\supset(\neg(\exists x)\neg Fx\wedge(\exists x)\neg Fx)  & \{1\}(7)\ \mathrm{D}\\
\{1\}   & (9)\ \neg(\forall x)Fx  & (8)\ \mathrm{TF}
\end{array}
\]
\iffalse
We then extended the rules to include existential generalization and existential instantiation which allow us to mirror common informal forms of argument involving the existential quantifier. The following gives an example of their use.
\begin{center}
$\{(\forall x) ((\exists y) Lxy \supset (\forall z) Lzx), (\exists x)(\exists
y) Lxy \}$ implies $(\forall v)(\forall z) Lvz.$
\end{center}
\[
\begin{array}{lll}
\{1\}   & (1)\  (\exists x)(\exists y) Lxy &  \mathrm{P}\\
\{1,2\}   & (2)\ (\exists y) Lwy  & (1)w\ \mathrm{EII}\\
\{3\}   & (3)\ (\forall x) ((\exists y) Lxy \supset   & 
\mathrm{P}\\
  &\ \ \ \  (\forall z) Lzx)  & \\
\{3\}   & (4)\ (\exists y) Lwy \supset   & (3)\ \mathrm{UI}\\
  &\ \ \ \ (\forall z) Lzw & \\
\{1,2,3\}   & (5)\ (\forall z) Lzw  & (2)(4)\ \mathrm{TF}\\
\{1,2,3\}   & (6)\ Lvw  & (5)\ \mathrm{UI}\\
\{1,\not 2,3\}   & (7)\ (\exists y) Lvy  & (5)\ \mathrm{EG};\{2\}\
\mathrm{EIE}\\ 
\{3\}   & (8)\ (\exists y) Lvy \supset (\forall z) Lzv  & (3)\ \mathrm{UI}\\
\{1,3\}   & (9)\  (\forall z) Lzv & (7)(8)\ \mathrm{TF}\\
\{1,3\}   & (10)\  (\forall v)(\forall z) Lzv & (9)\ \mathrm{UG}
\end{array}
\]
\newpage
Finally, we added rules for deriving schemata involving the identity predicate and illustrated their use with the following deduction.
\begin{center}
$\{(\forall x) Rxx, \neg (\forall x)(\forall y) Rxy \}$
implies $\neg (\exists x)(\forall y) x = y$.
\end{center}
\[
\begin{array}{lll}
\{1\}   & (1)\ (\forall x) Rxx  & \mathrm{P}\\
\{2\}   & (2)\ \neg (\forall x)(\forall y) Rxy  & \mathrm{P}\\
\{3\}   & (3)\ (\exists x)(\forall y) x = y  & \mathrm{P}\\
\{3,4\}   & (4)\ (\forall y) u = y  & (3)u\ \mathrm{EII}\\
\{1\}   & (5)\ Ruu  & (1)\ \mathrm{UI}\\
\{3,4\}   & (6)\ u=y  & (4)\ \mathrm{UI}\\
\{\}   & (7)\ u=y \supset (Ruu \equiv Ruy) & \mathrm{III}\\
\{3,4\}   & (8)\ u=x  & (4)\ \mathrm{UI}\\
\{\}   & (9)\ u=x \supset (Ruy \equiv Rxy) & \mathrm{III}\\
\{1,3,   & (10)\ Rxy  & (5)(6)\ \mathrm{TF};\\
\not4\} &  & (7)(8)\ \{4\}\ \mathrm{EIE} \\
 & & (9)\\
\{1,3\}   & (11)\ (\forall y)Rxy  & (10)\ \mathrm{UG}\\
\{1,3\}   & (12)\ (\forall x)(\forall y)Rxy  & (11)\ \mathrm{UG}\\
\{1,2,3\}   & (13)\ p \wedge \neg p  & (2)(12)\ \mathrm{TF}\\
\{1,2\}   & (14)\ (\exists x)(\forall y) x = y  \supset
& \{3\}(13)\ \mathrm{D}\\
 & (p \wedge \neg p) & \\
\{1,2\}   & (15)\ \neg (\exists x)(\forall y) x = y  & (14)\ \mathrm{TF}
\end{array}
\]
\fi