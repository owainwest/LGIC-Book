\subsection{Semantics}

We now introduce \emph{structures} as interpretations of monadic schemata. These play the role that truth-assignments played in the context of truth-functional logic, in that they bridge the gap between the syntactic objects of our (newly strengthened) language and their truth-values.

In order to specify a structure $A$ for a schema $S$ we need to
\begin{itemize}
\item specify a nonempty set $U^A,$ the universe of $A$;
\item specify sets $F^A, G^A, \ldots $ each of which is a subset of $U^A$ as
the extensions of the monadic predicate letters which occur in $S$;
\item specify an element $a \in U^A$ to assign to the variable $x,$ if $x$
occurs free in $S.$ 
\end{itemize}

\begin{aside}
    Item (1) specifies a \emph{universe of discourse}, that is, a collection of objects over which our variables of quantification range. % all the objects that are under our consideration (ie, all the objects we're referring to when we quantify). 
    Item (2) specifies the extension of each monadic predicate letter occurring in any schema under consideration. %for which elements of $U^A$ have the properties $F^A$, $G^A$, etc (or, equivalently, which $x \in U^A$ are such that $x \in F^A$, etc). 
    Item (3) makes sure that we assign definite values to free variables, if there are any, so that we can evaluate our sentence's truth. 
\end{aside}

When the variable $x$ has no free occurrences in the schema $S$, we write $A \models S$ as shorthand for ``the schema $S$ is true in the
structure $A,$'' alternatively ``the structure $A$ satisfies the schema $S$.'' Otherwise, we write $A\models S[a]$ as shorthand for ``the structure $A$ satisfies the schema $S$ relative to the assignment of $a$ to the variable $x$.'' 


\subsubsection*{Validity, satisfiability, implication, and equivalence}

We extend the notions of validity, satisfiability, implication, and equivalence to (closed) monadic quantificational schemata. 
\begin{itemize}
\item
A monadic schema $S$ is \emph{ valid} if and only if for every structure $A, A
\models S.$
\item
A monadic schema $S$ is \emph{ satisiable} if and only if for some structure $A,
A \models S.$
\item
A monadic schema $S$ \emph{ implies} a monadic schema $T$ if and only if for
every structure $A,$ if $A \models S,$ then $A \models T.$
\item
Monadic schemata $S$ and $T$ are equivalent if and only if $S$ implies $T$, and $T$ implies $S$. 
\end{itemize}

\begin{aside}
    A schema being valid means it is true in all possible interpretations, or as some would say, ``all possible worlds''. A schema is satisfiable if it's true in at least one possible world. 

    %Show that if $S$ implies $T$, $\mathbb{S}$ is the set of structures $S$ is true in, and $\mathbb{T}$ is the set of structures that $T$ is true in, then $\mathbb{S} \subseteq \mathbb{T}$.  
\end{aside}