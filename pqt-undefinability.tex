\subsection{Undefinability}
\subsubsection*{Cantor's Theorem and Cardinality Arguments}
We will show that for every infinite structure $C$ there is a subset $X\subseteq U^C$ which is \emph{not} definable over $C$. This result is a corollary to the celebrated Cantor Diagonal Theorem.
\begin{theorem}[Cantor]\label{cantordiag-thm}
Let $U$ be an infinite set and let $V_1, V_2, \ldots$ be a sequence of subsets of $U$. There is subset $W$ of $U$ such that for all $i\geq 1$, $W\neq V_i$.
\end{theorem}
\emph{Proof}: Suppose $U$ is an infinite set. Let $U^*= \{a_1, a_2, \ldots\}$ be a countably infinite subset of $U$ and let $V_1, V_2, \ldots$ be a sequence of subsets of $U$. Let $W=\{i\mid a_i\not\in V_i\}$. Note that for every $i$, $a_i\in W$ if and only if $a_i\not\in V_i$. It follows that for all $i$, $W\neq V_i$. \qed

\begin{aside}
    The idea in the above proof is to show that, regardless of which way we list the subsets $V_i$ of $U$, there will always be some other subset $W$ of $U$ which is not in the list. We construct $W$ by making sure it differs from each $V_i$ by at least one element; to do this, it suffices to let $a_i \in W$ iff $a_i \not \in V_i$. 
\end{aside}

In order to apply Theorem \ref{cantordiag-thm} to questions about definable sets we require the following result.
\begin{theorem}\label{countschema-thm}
For every structure $C$, there is a sequence $V_1,V_2,\ldots$ of subsets of $U^C$ such that for every set $X$ definable over $C$, there is an $i$ such that $X=V_i$. 

%For those who know a little bit of set theory, this says that there are only countably many definable subsets of any given structure.
\end{theorem}
\emph{Proof}: Every schema is a finite sequence of symbols drawn from a finite alphabet. Thus, we may arrange all schemata $S(x)$ in a list $S_1(x), S_2(x),\ldots$, first ordered by length, and then within length, alphabetically. We obtain a list $V_1,V_2,\ldots $ of all the sets definable over $C$ by setting $V_i=S_i[C]$ for all $i$. \qed

Theorem \ref{countschema-thm} entails that we can list all definable subsets of an infinite structure $C$, and Theorem \ref{cantordiag-thm} entails that no list can exhaust all the definable subsets of an infinite set. So we have our result:

\begin{corollary}
For every infinite structure $C$ there is a subset $X\subseteq U^C$ which is not definable over $C$.
\end{corollary}

\subsubsection*{The Compactness Theorem and Automorphisms of ``Non-standard Models''}

Of course, this gives us no idea which particular sets are not definable over a given infinite structure. In the case of the graph $B$ introduced above, we will show that if a set is neither finite nor co-finite, it is \emph{not} definable over $B$. In order to establish this, we will deploy one of the fundamental properties of polyadic quantification theory: \emph{compactness}. First, some definitions required to state the Compactness Theorem for Polyadic Quantification Theory.

\begin{definition}
A schema $S$ is \emph{satisfiable} if and only if for some structure $A$, $A\models S$.
\end{definition}

\begin{definition}
A set of schemata $\Gamma$ is \emph{satisfiable} if and only if there is structure $A$ such that for every schema $S\in \Gamma$, $A\models S$.
\end{definition}

\begin{definition}
A set of schemata $\Gamma$ is \emph{finitely satisfiable} if and only if for every finite set $\Delta\subseteq\Gamma$, $\Delta$ is satisfiable.
\end{definition}

\begin{theorem}[Compactness Theorem]\label{compact-thm}
For every set $\Gamma$ of schemata  of polyadic quantification theory, if $\Gamma$ is finitely satisfiable, then $\Gamma$ is satisfiable. 
\end{theorem}

Though the Compactness Theorem makes no mention of the notion of a derivation, one of its well-known proofs proceeds via the elaboration of a sound and complete formal system for logical deduction. We will discuss this development in Section \ref{pqt-proof-subsec}. For the moment, let's see how we can apply the Compactness Theorem to complete the analysis of the definable subsets of the structure $B$ specified above.
\begin{theorem}
If $V\subseteq U^B$ is definable over $B$, then $V$ is finite or $V$ is co-finite.
\end{theorem}
\emph{Proof}:
Suppose toward a contradiction that a schema $T(x)$ defines a set $V$ which is neither finite nor co-finite over $B$. Let $\Lambda = \{ S\mid B\models S\}$; $\Lambda$ is the set of all schemata true in the structure $B$ and is often called the \emph{complete theory} of $B$. Let $y$ and $z$ be fresh variables which occur nowhere in $T(x)$, or any of the schemata $S^n(x)$ for $n\geq 0$ (recall that $S^n(x)$ says that $x$ is the $n^{th}$ successor of the unique element with no predecessors). 

Define the set of schemata $\Gamma$ as follows.
\[
\Gamma = \Lambda\cup\{y\neq z \land T(y) \land \neg T(z)\}\cup\{\neg S^n(y) \land \neg S^n(z)\mid n\geq 0\}.
\]

Let $\Delta$ be a finite subset of $\Gamma$. As both $T[B]$ and $\neg T[B]$ are infinite by hypothesis, $\Delta$ can be satisfied by $B$ with suitable assignments from $U^B$ to the variables $y$ and $z$. Hence, by the Compactness Theorem, $\Gamma$ itself is satisfiable. Of course, if the structure $C$ satisfies $\Gamma$, then $C$ is not isomorphic to $B$ since the the elements of $U^C$ assigned to $y$ and $z$ in $C$ (call them $a$ and $b$ respectively) are not reachable in $C$ from the unique element of $C$ with no predecessor (whereas every element $b \in B$ is reachable in this manner). 

We will show that there is an automorphism $h$ of $C$ with $h(a)=b$. This will yield the desired  contradiction, since $C\models T(y|a)$ and $C\models \neg T(z|b)$. 

Note that $B$, and hence $C$, satisfy the following schemata.
\begin{itemize}
\item
$(\exists x)(\forall y)((\forall z)\neg Lzy \equiv x=y)$
\item
$(\forall x)(\exists y)(\forall z)(Lxz\equiv z=y)$
\item
$(\forall x)(\forall y)(\forall z)((Lxz\wedge Lyz)\supset x=y)$
\item
$(\forall x)\neg Lxx\\
\vdots\\
(\forall x)(\forall y_1)\ldots(\forall y_n)\neg Lxy_1\wedge Ly_1y_2\ldots\wedge Ly_nx\\
\vdots$
\end{itemize} 
The first three schemata guarantee that $L^C$ is an injective functional relation which is ``almost'' surjective -- there is a unique element of $U^C$ which lacks a pre-image under the function whose graph is $L^C$. Note that this guarantees that $U^C$ is infinite. 

\begin{aside}
    Why does this ensure that $U^C$ is infinite?
\end{aside}

The final infinite list of schemata guarantee that the the function whose graph is $L^C$ contains no finite cycles. Since $C$ is not isomorphic to $B$, all this implies that $C$ consists of an $L^C$ chain that is isomorphic to $B$ and a non-empty set of $L^C$ chains each of which is isomorphic to $\mathbb{Z}$ (the set of all integers) equipped with its usual successor relation. But, since $a$ and $b$ must lie on one or two of these ``$\mathbb{Z}$-chains,'' there is an automorphism $h$ of $C$ with $h(a)=b$ (if they lie on a single $\mathbb{Z}$-chain, shifting the $\mathbb{Z}$-chain works as an automorphism, whereas if they lie on two $\mathbb{Z}$-chains, interchanging the $\mathbb{Z}$-chains suffices). \qed