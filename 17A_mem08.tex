\subsection{Homomorphisms and monadic similarity: the central lemma}

The next lemma provides a useful sufficient condition for monadic similarity.
\begin{lemma}\label{hom-lem}
Let $A$ and $B$ be structures. If there is a homomorphism from $A$
onto $B$, then $A$ is monadically similar to $B.$ 
\end{lemma}
\emph{Proof}:
Let $A$ and $B$ be structures and suppose that $h$ is a homomorphism of $A$
onto $B.$ It suffices to show that for every simple monadic schema $S,$
\[A\models S\ \ \mbox{if and only if}\ \ B\models S,\]
since every pure monadic schema is a truth-functional compound of simple
monadic schemata.

We begin by observing that for every $c \in U^A$ and every one variable open
schema $S,$ $A$ makes $S$ true with respect to the assignment of $c$ to
``$x,$''
 if and only if $B$ makes $S$ true with respect to the assignment of
$h(c)$ to ``$x.$'' This follows immediately from the fact that $h$ is a
homomorphism.

Consider the simple schema $S$ and suppose that $S$ is the existential
quantification of the the one variable open schema $T$ (the case of universal
quantification is treated similarly).
Suppose $A \models S.$ Then, for some $c \in U^A$,
$A$ makes $T$ true with respect to the assignment of $c$ to
``$x.$'' It follows that $B$ makes $T$ true with respect to the assignment of $h(c)$
to ``$x.$'' Hence, $B \models S.$

Conversely, suppose $B \models S.$
Then, for some $c \in U^B$,
$B$ makes $T$ true with respect to the assignment of $c$ to
``$x.$'' Since $h$ is surjective, there is a $d \in U^A$ with $h(d) = c.$
It follows at once that $A$ makes $T$ true with respect to the assignment of
$d$ to ``$x.$'' Hence, $A \models S.$ \qed
\subsection{Types and monadic similarity}

We recall our discussion of element types:
\begin{itemize}
\item $T_1(x): Fx\wedge Gx$
\item $T_2(x): Fx\wedge \neg Gx$
\item $T_3(x): \neg Fx\wedge Gx$
\item $T_4(x): \neg Fx\wedge \neg Gx$
\end{itemize}
We say that a structure {\em realizes} a given type $T_i$ just in case it
makes the existential simple schema $(\exists x)T_i$ true.
\begin{example}\label{kinds-ex}
The following structure realizes all four of the types listed above.
\[A: U^A = \{1,2,3,4\}, F^A =\{1,3\}, G^A=\{1,2\}\]
Moreover, the 14 proper substructures of $A$ realize exactly the fourteen
proper nonempty subsets of the types listed above.
\end{example}
Lemma \ref{hom-lem} yields a useful necessary and sufficient condition for monadic similarity. 
\begin{lemma}\label{types-lem}
$A$ and $B$ realize the same types if and only if they are
monadically similar.
\end{lemma} 
\emph{Proof}:
If $A$ and $B$ realize the same types, then there is a single
structure $C$ which is a surjective homomorphic image of both $A$ and $B.$
Therefore, by our earlier result, $A$ is monadically similar to $C$ and $B$ is
monadically similar to $C.$ It follows at once that $A$ is monadically similar
to $B.$ The reverse implication follows immediately from the fact realization of a type is expressed by a pure monadic schema. \qed
\subsection{The small model theorem and the decidability of satisfiability}

Theorem \ref{smm-thm} is an immediate corollary to Lemma \ref{types-lem}.

\emph{Proof}(of Theorem \ref{smm-thm}):
It follows at once from Lemma \ref{types-lem} and Example \ref{kinds-ex}, that there is a collection $X$ of 15 structures each of size $\leq 4$
such that for any pure monadic schema $S$ involving only the predicate letters
``$F$'' and ``$G$,'' if $S$ is satisfiable, then there is a structure $A \in X$
such that $A \models S.$ 
More generally, there is a collection $X$ of $2^{(2^n)}-1$ structures each of size $\leq 2^n$
such that for any pure monadic schema $S$ involving only the predicate letters
``$F_1,$''$\ldots$ ``$F_n,$'' if $S$ is satisfiable, then there is a structure $A \in X$
such that $A \models S.$ \qed
\begin{corollary}
There is a decision procedure to determine whether a pure
monadic schema is satisfiable.
\end{corollary}
\subsection{The small model theorem and the decidability of satisfiability: an elaboration}

We elaborated the proof of the Small Model Theorem. Again, we focused on the case of schemata involving only the monadic predicate letters $F$ and $G$. We drew pictures, in ``Types View'', of 15 structures $A_1,\ldots,A_{15}$ each with universe of discourse included in $\{1,2,3,4\}$ and no two with the same universe of discourse.  Recall the element types:
\begin{itemize}
\item $T_1(x): Fx\wedge Gx$
\item $T_2(x): Fx\wedge \neg Gx$
\item $T_3(x): \neg Fx\wedge Gx$
\item $T_4(x): \neg Fx\wedge \neg Gx$
\end{itemize}
We constructed the structures $A_i$ by letting $j$ realize the type $T_j(x)$ for each $j\in U^{A_i}$. Let $A$ be an arbitrary structure. It is clear from our construction that there is an $i$ such that $A_i$ realizes exactly the same types as $A$. Moreover, since $A_i$ has exactly one element realizing any type that it realizes, $A_i$ is a surjective homomorphic image of $A$. It follows at once from the result of our last class that $A$ is monadically similar to $A_i$, that is, they satisfy exactly the same set of pure monadic schemata. Having thus concluded the proof, we turned to deriving a list of useful corollaries.
\begin{corollary}\label{monad-cor}
\begin{enumerate}
\item
For every schema $S$, if $S$ is satisfiable, then there is an $1\leq i\leq 15$ such that $A_i\models S$. 
\item
There is an algorithmic decision procedure to determine whether a schema $S$ is satisfiable.
\item\label{imp-item}
Schema $S$ implies schema $T$  if and only if 
\[\{i\mid A_i\models S\ \mbox{and}\ 1\leq i\leq 15\}\subseteq\{i\mid A_i\models T\ \mbox{and}\ 1\leq i\leq 15\}.\]
\item\label{equiv-item}
Schemata $S$ and $T$ are equivalent if and only if 
\[\{i\mid A_i\models S\ \mbox{and}\ 1\leq i\leq 15\}=\{i\mid A_i\models T\ \mbox{and}\ 1\leq i\leq 15\}.\]
\end{enumerate}
\end{corollary}

