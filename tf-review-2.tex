\subsection{Review}
\begin{mdframed}[linewidth=1]
\section*{Concept Review}
\textbf{Definitions}
\begin{itemize}
    \item $\mathbb{A}(X)$ is the set of all truth assignments over $X$. 
    
    \item $\mathbb{P}_X(S) = \{A | A \in \mathbb{A}(X) \text{ and } A \models S\}$ is \emph{the proposition expressed by S}. It's the set of truth assignments that satisfy $S$ (where truth assignments are restricted to those for sentence letters in the set $X$). 
    
    \item A schema $S$ \emph{implies} a schema $T$ iff for all truth-assignments $A$, if $A \models S$ then $A \models T$. In other words, $S$ implies $T$ iff the proposition expressed by $S$ is a subset of the proposition expressed by $T$. 
    
    \item A schema $S$ is \emph{equivalent} to a schema $T$ iff $S$ and $T$ are satisfied by exactly the same truth assignments (for all $A$, $A \models S$ iff $A \models T$). In other words, $S$ and $T$ are equivalent if they express the same proposition. 

    \item A list of TF-schemata is called \emph{succinct} iff no two schemata on the list are equivalent. 
    
    \item The \emph{power} of a schema $S$ is the length of the longest succinct list of schemata which $S$ implies. %In other words, it's the number of nonequivalent schema which $S$ implies. 
\end{itemize}

\textbf{Fun With Counting}
There are $n!$ ways to order a list of $n$ items. To see why, note that there are $n$ choices for the first element, $n - 1$ for the second, $n - 2$ for the third, resulting in $n(n - 1)(n - 2)...(1) = n!$ orderings. 

There are $\oc{n}{k} := \frac{n!}{(n-k)!}$ ways to pick an ordered list of $k$ elements from $n$ elements, $k \leq n$. As before, there are $n$ choices for the first thing, $n-1$ for the second, all the way down to $n-k+1$ for the $k^{th}$. This gives us the answer $\prod_{i = n - k + 1}^ni = \prod_{i = 1}^ni / \prod_{i = 1}^{n - k}i = \frac{n!}{(n-k)!}$   

There are $\binom{n}{k} := \frac{n!}{(n-k)!k!}$ ways to pick a subset of $k$ elements from $n$ elements, $k \leq n$. There are $\oc{n}{k}$ ordered lists of size $k$ from $n$. Since each subset of size $k$ corresponds to $k!$  of those ordered lists, we divide out by $k!$ to get $\frac{n!}{(n-k)!k!}$, for which we use the notation $\binom{n}{k}$, read as ``$n$ choose $k$''. 

\textbf{Expressive Completeness}

For any (arbitrary) proposition, there is a truth-functional schema which expresses that proposition. We noted that a schema can pick out individual truth-assignments by conjoining literals for each of the sentence letters (for example, the truth assignment $A_1$ which maps $p = \top, q = \top, r = \top$ is picked out by the sentence $(p \land q \land r)$). Sentences of this form are called \emph{terms}. We further noted that a disjunction of such terms (one for each truth-assignment in our proposition) was sufficient to express any proposition. 

\textbf{Power}

Suppose we have a sentence $S$ over $n$ sentence letters which is satisfied by $k$ truth assignments. Then the power of $S$ is $2^{2^n - k}$. To see why this is the case, note that there are $2^n$ truth assignments for $n$ sentence letters. If $S$ is satisfied by $k$ truth assignments, then those truth assignments must also satisfy $T$,  if $S$ implies $T$. So we can't ``choose'' whether or not to include any of those $k$ truth-assignments in the proposition expressed by $T$, because $\mathbb{P}_X(T)$ must include them. So we are left with $2^n - k$ truth-assignments, and since each of these $2^n - k$ truth assignments can be either in or out of the proposition expressed by $T$, the power of $S$ is then $2^{2^n - k}$.
\end{mdframed}

\newpage
\begin{mdframed}[linewidth=1]
\section*{Problems}
For the following problems, unless otherwise specified, let $X = \{p_1, p_2, p_3, p_4\}$ be the set of sentence letters under consideration. 
\begin{enumerate}
    \item What is the power of $p_1 \equiv p_2$?

    \item For four sentence letters as above, what is the length of the longest succinct list of schemata no two of which have the same power?

    \item What is the length of the longest succint list of schemata (from four sentence letters) each having power 256?

    \item What is the largest $n$ such that the conjunction of any two schema of power $n$ (with 4 sentence letters) is satisfiable?

    \item How many ways can you choose 3 marbles from a bag of 15 marbles, assuming the marbles are all distinct? How many ways to take out all 15 marbles from the bag, one by one? 

    \item How many ways are there to arrange $10$ people around a circular table, if we don't count rotations of the same order as being different?

    \item Is there a schema of power 22? If so, give one. If not, explain why it's not possible.

    \item How many non-equivalent schema over four letters have power greater than 256?
\end{enumerate}
\end{mdframed}

\newpage
\begin{mdframed}[linewidth=1]
\section*{Solutions}
\begin{enumerate}
    \item $2^8 = 256$. For four sentence letters, $p_1 \equiv p_2$ has $2^3 = 8$ satisfying truth assignments. To see why this is the case, note that given a choice for $p_1$, $p_2$ is fixed. So we have two choices for $p_1$, one choice for $p_2$, and two choices each for $p_3$ and $p_4$.

    Plugging in to our formula, we find that the power is $2^{2^4 - 8} = 2^{16 - 8} = 2^8$. 

    \item 17. The power of a sentence $S$ on $n$ sentence letters with $k$ satisfying truth assignments is $2^{2^n - k}$. $k$ can take any value from $0$ through $16$ inclusive when $n = 4$ (since we have $2^4 = 16$ truth-assignments), meaning that the power can be any one of $2^{16}, 2^{15}, ... , 2^0$.

    \item $\binom{16}{8}$. A schema on four sentence letters has power $256 = 2^8$ when it is satisfied by $8$ truth assignments (because $2^{2^4 - 8} = 2^8$). Since we have $2^4 = 16$ total truth assignments, the number of non-equivalent propositions of size 8 is the number of subsets of size 8 from 16, which is $\binom{16}{8}$. 

    \item $n = 2^7 = 128$. With four sentence letters, we have $2^4 = 16$ truth assignments. A schema has power $2^7$ is satisfied by $16 - 7 = 9$ truth assignments. By the pigeonhole principle, two schemata of power $2^7$ (hence both satisfied by 9 things) must have some satisfying truth-assignment in common (because $9 + 9 = 18 > 16$). Hence the conjunction of any two schemata of power $2^7$ is satisfiable, because there must be a truth assignment that satisfies them both. 

    Note that $2^7$ is the highest power that works, because being satisfied by less than 9 truth-assignments (therefore having a greater power) would mean that the two sentences need not have a satisfying truth-assignment in common. For example, if both sentences were satisfied by 8 truth assignments each, those sets of satisfying truth-assignments could be disjoint, hence the conjunction of the two sentences would not be satisfiable. 

    \item $\binom{15}{3}$, $15!$

    \item $9!$. There are $10!$ ways to order $10$ people around the table, but that considers different rotations of the same order as different seating arrangements. Since there are 10 rotations of any such ordering, we divide $10!$ by $10$, giving us the answer $9!$.

    \item No. The power of a schema is always some power of 2. 22 is not a power of 2. 

    \item $\sum_{i = 0}^7\binom{16}{i}$. We have $2^4 = 16$ total truth-assignments. The power of a schema $S$ on four sentence letters is greater than $256 = 2^8$ when $S$ is satisfied by less than $8$ truth-assignments (because our formula for power is $2^{2^n - k}$ with $n=4$ in this case, hence power is greater than $2^8$ when $k$ is less than 8). Hence our answer equal to the number of schema that express a proposition of size 0, plus the number that express a proposition of size 1.... plus the number that express a proposition of size 7. Remember that $\binom{n}{k}$ represents the number of size-$k$ subsets from $n$ things, and since propositions are simply subsets of truth-assignments, we arrive at our answer $\sum_{i = 0}^7\binom{16}{i}$.

\end{enumerate}
\end{mdframed}