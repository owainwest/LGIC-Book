\subsection{Basic Semantics of Truth-Functional Logic}
Given a truth-functional schema like $((p \wedge q) \vee r)$, we cannot determine whether the schema is true or false unless we know whether $p$, $q$, and $r$ are true or false. That is, any schema requires a truth-assignment to its sentence letters before it can be evaluated. 

\begin{definition}[Truth-assignment]
Let $X$ be a set of sentence letters. A \emph{truth-assignment} $A$ for $X$ is a mapping which associates with each sentence letter $q\in X$ one of the two truth values $\top$ or $\bot$; we write $A(q)$ for the value that $A$ associates to $q$. 
\end{definition}

\begin{definition}
Suppose $S$ is a truth-functional schema such that every sentence letter with an occurrence in $S$ is a member of $X$. We say a truth assignment $A$ for $X$ \emph{satisfies} such a schema $S$ (and write $A\models S$) if and only if $S$ receives the value $\top$ relative to the truth assignment $A$. 
\end{definition}

\begin{example}
Take the schema $S = ((p \wedge q) \vee r)$, with truth assignment $A$ such that $A(p) = \top$, $A(q) = \bot$, and $A(r) = \bot$, we have that $S$ receives the value $\bot$. In other words $A$ does not satisfy $S$. ($A \not \models S$).
\end{example}

\subsubsection*{Interpreting the Material Conditional}
Let's return to our potential lovers and restrict attention to just two of them, 1 and 2. How could express the statement that all love is requited among these two sweethearts? The natural mode of expression is: if 1 loves 2, then 2 loves 1, and if 2 loves 1, then 1 loves 2. This is a perfect candidate for using the material conditional.

Using the sentence letters $p_{11}, p_{12}, p_{21}, p_{22}$ as earlier interpreted, we can express the happy state that all love among 1 and 2 is requited by the schema
\[ 
    R: (p_{12}\supset p_{21}) \wedge (p_{21}\supset p_{12})
\]
or, equivalently, 
\[
    p_{12} \equiv p_{21}
\]

\begin{aside}
    In how many of the possible love scenarios among 1 and 2 is all love requited? Count the number of satisfying truth-assignments to $R$!
\end{aside}

While the motivations for the truth-functional definitions for the other connectives normally seem evident to new logicians, the material conditional often gives people trouble. Let's consider generalized conditionals as a route to motivating the truth-functional interpretation of the conditional offered above. Of course, the statement ``if an integer is divisible by six, then it is divisible by three,'' is true, and thence each of the following statements, which are instances of this
general statement, are true.
\begin{itemize}
\item ``If twelve is divisible by six, then twelve is divisible by three.''
\item ``If three is divisible by six, then three is divisible by three.''
\item ``If two is divisible by six, then two is divisible by three.''
\end{itemize}
%Note that the preceding sentences are in the form $\top \supset \top$, $\bot \supset \top$, and $\bot \supset \bot$ respectively. 

Therefore, if the conditional involved is to be understood truth-functionally, then its interpretation must satisfy the conditions imposed by the first, third, and fourth rows of the material conditional's truth-table. On the other hand, the falsity of the conditional ``if twelve is divisible by six, then twelve is divisible by seven,'' mandates the condition imposed by the second row of the truth-table.

\subsubsection*{An Inductive Proof}
Let's do a simple inductive proof about truth-functional satisfaction, as an illustration of the use of mathematical induction, especially in application to reasoning about truth-functional schemata.

\begin{proposition}\label{parity-prop}
For every $n\geq 2$ and every set $X=\{q_1,\ldots,q_n\}$ of $n$ distinct sentence letters, a truth assignment $A$ for $X$ satisfies the schema
\[S_n: (\ldots(q_1\oplus q_2)\ldots\oplus q_n)\]
if and only if $A$ assigns an odd number of the sentence letters in $X$ the value $\top$.
\end{proposition}

\begin{proof}
    We prove the proposition by induction on $n$. 
    \begin{itemize}
        \item Basis: Examination of the truth table for $\oplus$ suffices to establish the proposition for the case $n=2$.

        \item Induction Step: Suppose the proposition holds for a number $k\geq 2$, that is, 
        for every truth assignment $A$ for $\{q_1,\ldots,q_k\}$, $A\models S_k$ if and only if $A$ assigns an odd number of the sentence letters in $\{q_1,\ldots,q_k\}$ the value $\top$; this is our induction hypothesis.
        We proceed to show that the proposition also holds for $k+1$. Let $A'$ be an assignment to the sentence letters 
        $\{q_1,\ldots,q_{k+1}\}$ and let $A$ be its restriction to $\{q_1,\ldots,q_k\}$. We consider two cases. First, suppose that $A'(q_{k+1}) = \top$. In this case, $A'\models S_{k+1}$ if and only if $A\not\models S_k$ if and only if (by our induction hypothesis) $A$ assigns an even number of the sentence letters $\{q_1,\ldots,q_k\}$ the value $\top$. Hence, if $A'(q_{k+1}) = \top$, then $A'\models S_{k+1}$ if and only if $A'$ assigns an odd number of the sentence letters in $\{q_1,\ldots,q_{k+1}\}$ the value $\top$. On the other hand, suppose that $A'(q_{k+1}) = \bot$. In this case, $A'\models S_{k+1}$ if and only if $A\models S_k$ if and only if (by our induction hypothesis) $A$ assigns an odd number of the sentence letters $\{q_1,\ldots,q_k\}$ the value $\top$. Hence, if $A'(q_{k+1}) = \bot$, then $A'\models S_{k+1}$ if and only if $A'$ assigns an odd number of the sentence letters in $\{q_1,\ldots,q_{k+1}\}$ the value $\top$. This concludes the proof, since either $A'(q_{k+1}) = \top$ or $A'(q_{k+1}) = \bot$. 
    \end{itemize}
\end{proof}

\subsubsection*{The Centrality of Satisfaction}

The satisfaction relation is the fundamental semantic relation. It is where language and the world meet; in the case to hand, language consists of truth-functional schemata and the possible worlds they describe are truth assignments to sentence letters. As the course progresses, we will encounter more textured representations of the world (relational structures) and richer languages to describe them (monadic and polyadic quantification theory). We now define some of the central notions of truth-functional logic in terms of satisfaction. These definitions will generalize directly to the more textured structures and richer languages we encounter later. 

For the following definitions, we suppose that $S$ and $T$ are truth-functional schemata and that $A$ ranges over truth assignments to sets of sentence letters which include all those that occur in either $S$ or $T$.

\begin{definition}\label{tf-eq-sat-val-def}
$S$ \emph{implies} $T$ if and only if for every truth assignment $A$, if $A\models S$, then $A\models T$.
\end{definition}

\begin{definition}
$S$ is \emph{equivalent} to $T$ if and only if $S$ implies $T$ and $T$ implies $S$
\end{definition}

\begin{definition}
$S$ is \emph{satisfiable} if and only if for some $A$, $A\models S$.
\end{definition}

\begin{definition}
$S$ is \emph{valid} if and only if every truth assignment satisfies $S$. 
\end{definition}
\subsubsection*{Examples of equivalence and the material biconditional}

Try to see why the following are equivalent - either by appealing to your understanding of what the connective ``means'' or by going back to the truth tables. 

\begin{itemize}
\item $p\oplus q$ is equivalent to $q\oplus p$ (commutativity of exclusive disjunction)  
\item $(p\oplus q)\oplus r$ is equivalent to $p\oplus(q\oplus r)$ (associativity of exclusive disjunction).
\item $p \equiv q$ ius equivalent to $(p \supset q) \land (q \supset p)$
\end{itemize}

Note that both conjunction and inclusive disjunction are also commutative and associative, whereas the material conditional is neither. 

\begin{aside}
    Try to to think of examples of (binary) truth-functional connectives which are commutative but not associative, and associative but not commutative.
\end{aside}