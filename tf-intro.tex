\subsection{Introduction to Truth-Functional Logic}
Throughout the course we will see a few different systems for formalizing statements. Each consists of a formal language to represent statements, and a way to interpret the meaning of statements in that language. Truth-functionalF logic is the simplest of these systems we will learn. 

\subsubsection*{Components of Truth Functional Logic}
\begin{enumerate}
    \item Language (the \emph{Syntax})
    \begin{enumerate}
        \item sentence letters
        \item connectives
    \end{enumerate}
    \item Interpretation (the \emph{Semantics})
    \begin{enumerate}
        \item A function that assigns $\top$ or $\bot$ (true or false) to each sentence letter, called a \textbf{truth-assignment}
        \item Fixed \textbf{truth-functional semantics} for each connective
    \end{enumerate}
\end{enumerate}

\textbf{Sentence letters} such as $p, q, r, \ldots$ schematize statements (in natural language) which are true or false, and \textbf{connectives} such as $\wedge, \vee, \neg, \supset, \ldots$ are used to combine sentence letters into compound schemata. 

\begin{example}
We might say that $p$ represents the statement ``it is a Wednesday''. This would be reasonable, since it's definitely either true or false that today is Wednesday. On the other hand, ``is today Wednesday?'' isn't a statement, so we wouldn't encode it as a sentence letter. Truth-functional logic deals with the truth or falsity of \emph{statements} only. 
\end{example}