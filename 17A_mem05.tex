\subsection{Counting Selections}
Our leading question from the end of the last section brings us to an interlude on permutations and combinations: how many ways can we select $k$ members of a set of size $n$? There is an ambiguity here: are we counting modes of selection, which involve the order of choices, or collections of members selected, where the order of selection is irrelevant? Once we recognize the ambiguity, we can proceed to count both. We introduced notation for each: $\oc{n}{k}$ for the number of ordered sequences of $k$ distinct elements that can be drawn from a set of size $n$  and $\binom{n}{k}$ for the number of subsets of size $k$ that are included in set of size $n$. 

Let's first evaluate $\oc{n}{k}$, the number of ordered sequences of size $k$ you can pick from a set of size $n$. Suppose we think of counting the ways $n$ students could fill a row of length $k$ in a lecture hall. Let's suppose the seats are labelled $1, 2, \ldots, k$. There are $n$ choices for the student to fill seat 1; once that seat is filled, there are $n-1$ choices for the student to fill seat 2; and so on until there are $(n-k)+1$ choices for the student to fill seat $k$. Hence, by the product rule, there are $n\cdot(n-1)\cdots((n-k)+1)$ ways of filling all $k$ seats, that is, $\oc{n}{k}=n\cdot(n-1)\cdots((n-k)+1)$. 

Now that we have counted the number of ordered sequences, we can see how to count the number of subsets. By the same reasoning, each subset of size $k$ appears as the content of $k\cdot(k-1)\cdots2\cdot1$ ordered sequences of length $k$; this number is called $k$ \emph{factorial} and is often abbreviated as $k!$. Hence, 
\[\binom{n}{k} = \frac{\oc{n}{k}}{k!}.\]  
Observe that 
\[\oc{n}{k}=\frac{n!}{(n-k)!}\]
from which it follows that 
\[\binom{n}{k} = \frac{n!}{k!\,(n-k)!}.\]
This last formulation makes transparent a symmetry in the values of $\binom{n}{k}$, namely, for every $k$ between $0$ and $n$, $\binom{n}{k}=\binom{n}{n-k}$. This accords nicely with the observation that complementation induces a one-one correspondence between the subsets of size $k$ and the subsets of size $(n-k)$ that can be selected from a set of size $n$. Note also that it determines in a non-arbitrary way that the value of $0!$ is 1.

\begin{aside}
    Consider picking a panel of three students from a class of 10. How many ways can you do this? Is it the same as the number of ways you could pick 7 of the 10 students to \emph{not} be on the panel, using the non-picked students for the panel?
\end{aside}

Let's not forget how this all began. The length of the longest succinct list of schemata with power 32 is $\binom{8}{3}=56$.

\subsection{The length of an ``implicational anti-chain''}

We actually used our new found ability to count selections to answer a different question: Is there a sequence of seventy schemata $S_1,\ldots,S_{70}\in\mathbb{S}(X)$ such that for every $1\leq i\neq j\leq 70,$ $S_i$ does \emph{not} imply $S_j$? Such a sequence of schemata is called an \emph{implicational anti-chain} (of length 70).
As observed earlier, a schema $S\in\mathbb{S}(X)$ implies a schema $T\in\mathbb{S}(X)$ if and only if $\prop{S}{X}\subseteq\prop{T}{X}$. It follows that the answer to our question about an implicational anti-chain of length seventy will be the same as the answer to the following question about an anti-chain of length seventy with respect to the subset relation: Is there a list of seventy subsets of $\mathbb{A}(X)$, $P_1,\ldots,P_n$, such that for every $1\leq i\neq j\leq 70,$ $P_i$ is \emph{not} a subset of $P_j$? Note that if two finite sets, $P$ and $Q$, have the same number of members, and $P$ is not equal to $Q$, then $P$ is not a subset of $Q$ and $Q$ is not a subset of $P$. Therefore, if there are seventy distinct subsets of $\mathbb{A}(X)$ all of the same size, then the answer to our question is yes. Since $\mathbb{A}(X)$ has eight members, a positive answer to our question followed immediately by evaluating 
\[
\binom{8}{4}= \frac{8\cdot7\cdot6\cdot5}{4\cdot3\cdot2\cdot1}=70.
\] 

Note that our argument merely shows that there is an implicational anti-chain of length 70; it does not establish that there is no longer implicational anti-chain consisting of schemata in $\mathbb{S}(X)$. This is, indeed, true, but a more sophisticated argument is required to establish this result, which follows from the famous Sperner's Theorem. For a reference on Sperner's Theorem, see: 

Van Lint and Wilson, \emph{A course in combinatorics}, Chapter 6: Dilworth's theorem and extremal set theory.

\subsection{Truth-functional Satisfiablity: Is there an efficient decision procedure?}

We observed that the finitary character of the semantics for truth-functional logic immediately yields an algorithm to decide the satisfiability of schemata of truth-functional logic. In particular, suppose $S\in\mathbb{S}(X)$ for some finite set of sentence letters $X$. Note first that for each truth-assignment $A\in\mathbb{A}(X)$ there is a simple and efficient algorithm, call it $M$, to determine whether $A\models S$. Thus, in order to test the satisfiability of $S$, we need only list $\mathbb{A}(X)$ in some canonical order $A_1,\ldots, A_{2^{|X|}}$ and use $M$ to test whether the successive $A_i$ satisfy $S$. 


%\paragraph{But is there an efficient algorithm?}
Of course, this algorithm is not efficient, in the sense that it's running time is potentially exponential in the length of its input. The question whether there is an efficient algorithm to decide the satisfiability of truth-functional schemata is generally regarded as one of the most significant open mathematical problems of our time -- for further information visit: 

http://www.claymath.org/millennium-problems/p-vs-np-problem.

\newpage
\begin{mdframed}[linewidth=1]
\section*{Concept Review}
\textbf{Fun With Counting}
There are $n!$ ways to order a list of $n$ items. To see why, note that there are $n$ choices for the first element, $n - 1$ for the second, $n - 2$ for the third, resulting in $n(n - 1)(n - 2)...(1) = n!$ orderings. 

There are $\oc{n}{k} := \frac{n!}{(n-k)!}$ ways to pick an ordered list of $k$ elements from $n$ elements, $k \leq n$. As before, there are $n$ choices for the first thing, $n-1$ for the second, all the way down to $n-k+1$ for the $k^{th}$. This gives us the answer $\prod_{i = n - k + 1}^ni = \prod_{i = 1}^ni / \prod_{i = 1}^{n - k}i = \frac{n!}{(n-k)!}$   

There are ${n \choose k} := \frac{n!}{(n-k)!k!}$ ways to pick a subset of $k$ elements from $n$ elements, $k \leq n$. There are $\oc{n}{k}$ ordered lists of size $k$ from $n$. Since each subset of size $k$ corresponds to $k!$ orderings of those lists, we divide out by $k!$ to get $\frac{n!}{(n-k)!k!}$, which we denoted as $n \choose k$, read ``$n$ choose $k$''. 

\end{mdframed}

\newpage
\begin{mdframed}[linewidth=1]
\section*{Problems}
For the following problems, unless otherwise specified, let $X = \{p_1 p_, p_3, p_4\}$ be the set of sentence letters under consideration. 
\begin{enumerate}
    \item What is the power of $p_1 \equiv p_2$?

    \item For four sentence letters as above, what is the length of the longest succinct list of schemata no two of which have the same power?

    \item What is the length of the longest succint list of schemata (from four sentence letters) each having power 256?

    \item What is the largest $n$ such that the conjunction of any two schema of power $n$ (with 4 sentence letters) is satisfiable?

    \item Is the following sentence valid, satisfiable but not valid, or unsatisfiable?
    \[
        (a \equiv b) \supset (a \vee \lnot b)
    \]

    \item Valid, satisfiable, or unsatisfiable?
    \[
        (b \vee (b \supset a)) \land (\lnot b \vee (a \supset b))
    \]

    \item Valid, satisfiable, or unsatisfiable?
    \[
        (a \equiv b) \land (b \equiv c) \land (a \oplus b)
    \]

    \item How many truth-assignments for the given letters satisfy 
    \[
         (a \equiv b) \land (b \equiv c) \land (c \equiv d)
     \] 

     \item How many truth-assignments to the given letters satisfy
    \[
        (a \oplus b) \vee (b \oplus c) \vee (c \oplus d)
    \]

    \item How many ways can you choose 3 marbles from a bag of 15 marbles, assuming the marbles are all distinct? How many ways to take out all 15 marbles from the bag, one by one? 

    \item How many ways are there to arrange $10$ people around a circular table, if we don't count rotations of the same order as being different?

    \item Is there a schema of power 22? If so, give one. If not, explain why it's not possible.

    \item How many non-equivalent schema over four letters have power greater than 256?
\end{enumerate}
\end{mdframed}

\newpage
\begin{mdframed}[linewidth=1]
\section*{Solutions}
\begin{enumerate}
    \item $2^8 = 256$. For four sentence letters, $p_1 \equiv p_2$ has $2^3 = 8$ satisfying truth assignments. To see why this is the case, note that given a choice for $p_1$, $p_2$ is fixed. So we have two choices for $p_1$, one choice for $p_2$, and two choices each for $p_3$ and $p_4$.

    Plugging in to our formula, we find that the power is $2^{2^4 - 8} = 2^{16 - 8} = 2^8$. 

    \item 17. The power of a sentence $S$ on $n$ sentence letters with $k$ satisfying truth assignments is $2^{2^n - k}$. $k$ can take any value from $0$ through $16$ inclusive when $n = 4$ (since we have $2^4 = 16$ truth-assignments), meaning that the power can be any one of $2^{16}, 2^{15}, ... , 2^0$.

    \item $16 \choose 8$. A schema on four sentence letters has power $256 = 2^8$ when it is satisfied by $8$ truth assignments (because $2^{2^4 - 8} = 2^8$). Since we have $2^4 = 16$ total truth assignments, the number of non-equivalent propositions of size 8 is the number of subsets of size 8 from 16, which is $16 \choose 8$. 

    \item $n = 2^7 = 128$. With four sentence letters, we have $2^4 = 16$ truth assignments. A schema has power $2^7$ is satisfied by $16 - 7 = 9$ truth assignments. By the pigeonhole principle, two schemata of power $2^7$ (hence both satisfied by 9 things) must have some satisfying truth-assignment in common (because $9 + 9 = 18 > 16$). Hence the conjunction of any two schemata of power $2^7$ is satisfiable, because there must be a truth assignment that satisfies them both. 

    Note that $2^7$ is the highest power that works, because being satisfied by less than 9 truth-assignments (therefore having a greater power) would mean that the two sentences need not have a satisfying truth-assignment in common. For example, if both sentences were satisfied by 8 truth assignments each, those sets of satisfying truth-assignments could be disjoint, hence the conjunction of the two sentences would not be satisfiable. 

    \item This is valid. Note that if $a \equiv b$, $a \vee \lnot b$ holds (since one of $a$ or $\lnot b$ must be true, hence the consequent is true, hence the conditional is true). If $a$ is not equivalent to $b$, then the conditional holds because false implies anything. 

    \item Valid. If $b$ is true, then the left conjunct is clearly true. If $b$ is not true, then $b \supset a$ is true (false implies anything), hence the left conjunct is true as well. Similarly, if $b$ is true then the right conjunct is true (since both falsity and truth imply truth), and if $b$ is false then $\lnot b$ is true, hence the right conjunct is true again. 

    \item Unsatisfiable. Since $a$ is equivalent to $b$ and $b$ is equivalent to $c$, $a$ is equivalent to $c$ (in other words, $\equiv$ is \emph{transitive}). But $a$ can't be equivalent to $c$, because $a \oplus c$. So the sentence is unsatisfiable. 

    \item 2. Picking true/false for $a$ fixes the truth-values of the remaining letters. 

    \item 14. To get this answer, we note that there are 16 ($2^4$) truth assignments in total, count the number which do not satisfy our sentence, and subtract that number from 16. The sentence is only not satisfied when each of $a, b, c, d$ have the same truth-value, so there are 2 non-satisfying truth-assignments. This means there are $16 - 2 = 14$ satisfying truth assignments. 

    \item $15 \choose 3$, $15!$

    \item $9!$. There are $10!$ ways to order $10$ people around the table, but that considers different rotations of the same order as different seating arrangements. Since there are 10 rotations of any such ordering, we divide $10!$ by $10$, giving us the answer $9!$.

    \item No. The power of a schema is always some power of 2. 22 is not a power of 2. 

    \item $\sum_{i = 0}^7{16 \choose i}$. We have $2^4 = 16$ total truth-assignments. The power of a schema $S$ on four sentence letters is greater than $256 = 2^8$ when $S$ is satisfied by less than $8$ truth-assignments (because our formula for power is $2^{2^n - k}$ with $n=4$ in this case, hence power is greater than $2^8$ when $k$ is less than 8). Hence our answer equal to the number of schema that express a proposition of size 0, plus the number that express a proposition of size 1.... plus the number that express a proposition of size 7. Remember that $n \choose k$ represents the number of size-$k$ subsets from $n$ things, and since propositions are simply subsets of truth-assignments, we arrive at our answer $\sum_{i = 0}^7{16 \choose i}$.

\end{enumerate}
\end{mdframed}