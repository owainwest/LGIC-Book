\subsection{The Spectrum of a Schema}

We began to discuss another interesting aspect of the expressive power of polyadic quantification theory. We write $\mathbb{Z}^+$ for the set of positive integers $\{1,2,3,\ldots\}$. The \emph{spectrum} of a schema $S$ (written \spec{S}) is defined as follows.
\[
\spec{S}=\{n\in \mathbb{Z}^+\mid \modn{S}{n}\neq\emptyset\}.
\]
We can restate the definition in slightly different terms. Say that a schema $S$ \emph{admits} a positive integer $n$ if and only if there is a structure $A$ such that $A\models S$ and $\card{U^A}=n$. Then $\spec{S}$ is exactly the set of positive integers $n$ such that $S$ admits $n$. 
\subsection{Finite Sets and Co-finite Sets are Spectra}

Let $F$ be a finite set of positive integers. We asked, ``Is there a schema $S$ such that $\spec{S}=F$?'' We began with singletons and showed that for every positive integer $n$, there is a schema, call it $\mathsf{S}_n$  such that $\spec{\mathsf{S}_n}=\{n\}$. 
We may take $\mathsf{S}_n$ to be the following schema.
\[
(\exists x_1)\ldots(\exists x_n)\bigwedge_{1\leq i<j\leq n}x_i\neq x_j\wedge \neg(\exists x_1)\ldots(\exists x_{n+1})\bigwedge_{1\leq i<j\leq n+1}x_i\neq x_j
\]
It follows at once that for any finite set of positive integers  $F=\{n_1,\ldots,n_k\}$,
\[
\spec{\mathsf{S}_{n_1}\vee\ldots\vee\mathsf{S}_{n_k}}= F.
\]
Moreover, we noted that
\[
\spec{\neg(\mathsf{S}_{n_1}\vee\ldots\vee\mathsf{S}_{n_k})}= \mathbb{Z}^+ - F.
\]
Thus, every finite set of positive integers and the complement of every finite set of positive integers is a spectrum (the latter sets are called \emph{co-finite}).
\subsection{Complementation and the Spectrum Problem}

It is actually quite unusual that the spectrum of the negation of a schema $S$ is equal to the complement of the spectrum of $S$. We considered the following example.
Recall the schema $\sg\wedge\oner$ which defines the collection of 1-regular simple graphs. We reminded ourselves that we'd already noticed that $\spec{\sg\wedge\oner}$ is the set of even numbers, that is, $\spec{\sg\wedge\oner}=\{2i\mid i\in\mathbb{Z}^+\}$. On the other hand, $\spec{\neg(\sg\wedge\oner)}=\mathbb{Z}^+$. This behavior is actually typical. Later in the course we may be in a position to prove the following important fact: if the spectrum of a schema $S$ is neither finite nor cofinite, then the spectrum of the negation of $S$ is not equal to the complement of the spectrum of $S$. This led to a brief discussion of the question, ``Is there a schema $S$ such that the complement of the spectrum of $S$ is not the spectrum of any schema whatsoever?'' Nobody knows the answer to this question. It is known that a set of positive integers is a spectrum if and only if it is in the complexity class $\mathsf{NE}$, the set of problems solvable in non-deterministic (linear) exponential time on a Turing machine. For those of you who might like to learn more about this open problem, I've uploaded a paper ``Fifty Years of the Spectrum Problem'' to the course Canvas site.
\subsection{Further Examples of Infinite, Co-infinite Spectra}

We went on to modify the schema $\sg\wedge\oner$ to give an example of a schema whose spectrum is the set of odd numbers. The modified schema states the condition that there is an isolated node $w$, and every node other than $w$ has degree one, in addition to ensuring that any satisfying structure is a simple graph. 

We presented a more substantial example, a schema $S$ with $\spec{S}=\{k^2\mid k\in \mathbb{Z}^\}$. The schema involved a triadic predicate letter $H$ and a monadic predicate $F$. $S$ is the conjunction of the following schemata.
\begin{itemize}
\item $(\forall x)(\forall y)((Fx\wedge Fy)\supset(\exists z)(\forall w)(Hxyw\equiv w=z))$
\item $(\forall x)(\forall y)(\forall z)(Hxyz\supset (Fx\wedge Fy))$
\item $(\forall x)(\exists y)(\exists z)Hyzx$
\item $(\forall x)(\forall y)(\forall z)(\forall w)(\forall v)((Hxyv\wedge Hzwv)\supset (x=z\wedge y=w))$
\end{itemize}
Suppose $A\models S$. The conjunction of the first two schemata guarantee that $H^A$ is the graph of a binary function mapping $F^A\times F^A$ to $U^A$. Further conjoining the third and fourth schemata guarantee that this function is a bijection, thereby insuring that $\card{U^A}$ is a perfect square.
\iffalse 
On Wednesday, we (well you and Grace) looked at another important class of graphs, namely, equivalence relations, and saw how they can be put to use in generating schemata with a wide range of spectra. A graph $A$ is an \emph{equivalence relation} if and only if $L^A$ is reflexive, symmetric, and transitive, that is, if and only if $A\models\eqr$, where \eqr\ is the conjunction of the following schemata.
\begin{itemize}
\item
\refl: $(\forall x)Lxx$
\item
\sym: $(\forall x)(\forall y)(Lxy\supset Lyx)$
\item
\trans: $(\forall x)(\forall y)(\forall z)(Lxy\supset(Lyz\supset Lxz))$
\end{itemize}
Now suppose we'd like to construct a schema $S$ such that
\begin{itemize}
\item
$S$ implies \eqr, and
\item
$\spec{S}=\{3i+1\mid i\in\mathbb{Z}^+\cup\{0\}\}$.
\end{itemize}
The easiest way to meet the first condition is to formulate $S$ as a conjunction, one conjunct of which is \eqr\ itself. But what more should we say? Well, the universe $U^A$ of an equivalence relation $A$ is partitioned into mutually disjoint \emph{equivalence classes} by the relation $L^A$; for each $a\in U^A$, the equivalence class $\hat{a}$ of $a$, is $\{b\in U^A\mid \op{a}{b}\in L^A\}$. Now if we can construct a schema $T$ that says every equivalence class but one is of size three, and that the exceptional equivalence class is of size one, then we may take $S$ to be the conjunction of \eqr\ and $T$. The following schema $T$ does the job.
\begin{align*} 
(\exists x)&(\forall t)((\forall y) (Lty\supset y=t)\equiv x=t)\wedge  \\
&(\forall z)((\exists r)(r\neq z\wedge Lrz)\supset\\ &(\exists v)(\exists w)(v\neq z\wedge v\neq w\wedge w\neq z\wedge(\forall u)(Luz\equiv(u=z\vee u=v\vee u=w))))
\end{align*}
We generalized this to show that for every $j$ and $0\leq k<j$, there is a schema $S$ such that $S$ implies \eqr, and $\spec{S}=\{nj+k\mid n\in \mathbb{Z}^+\}$. 
\iffalse
In general it is not the case that $\spec{\neg S} = \mathbb{Z}^+ - \spec{S}$. Indeed, for the schema $S$ constructed above, $\spec{\neg S} = \mathbb{Z}^+.$ We observed that for every $n\in\mathbb{Z}^+$, $\spec{\neg(\exists^{=n}x)x=x} = \mathbb{Z}^+-\spec{(\exists^{=n}x)x=x}$. We discussed the fact that for every schema $S$, either $\spec{S}$ is finite, or $\spec{\neg S}$ is finite. Therefore, if $\spec{S}$ is infinite and co-infinite, then $\spec{\neg S}\neq\mathbb{Z}^+ - \spec{S}$. 


It is an open question, first posed by G\"{u}nter Asser in 1955, whether for every schema $S$, there is a schema $T$ such that $\spec{T}=\mathbb{Z}^+ - \spec{T}$. It turns out that this question is equivalent to the closure of the comlexity class $\mathsf{NE}$ (non-deterministic exponential time) under complementation, that is, ``$\mathsf{NE}=\mathsf{co\mbox{-}NE}$?''



We will go on to explore further examples of spectra. Note that in general it is not the case that $\spec{\neg S} = \mathbb{Z}^+ - \spec{S}$. Convince yourself by constructing some examples where the equation fails! Can you think of examples where the equation holds? Can you think of a general condition on $\spec{S}$ that guarantees the failure of the equation? We will discuss these questions in class on Monday. 
\fi
\fi