\subsection{Review}
\begin{mdframed}[linewidth=1]
\section*{Concept Review}
\textbf{Binary Relations}:
Given a set $S$, a binary relation $R$ between a domain $A$ and a codomain $B$ is a set of pairs $(a, b)$ where $a \in A$ and $b \in B$. We will normally consider relations whose domain equals their codomain. In this case, a relation $R$ over a set $S$ is a subset of $S^2 = S \times S = \{(s, s') | s, s' \in S\}$ (this is the ``Cartesian product'' of $S$ with itself, the set of all \emph{ordered pairs} of members of $S$). 
Notice that we have identified the interpretation of a dyadic predicate letter in a structure with a relation in this sense; this continues our practice of treating our schematic languages \emph{extensionally}.
For example, let $L$ be a dyadic predicate letter. The structure $A$ with $U^A=\{0,1,\ldots\}$ and $L^A=\{(i,j)\mid i,j\in\mathbb{N}\mbox{ and } i<j\}$ interprets $L$ as the strict linear ordering relation on the set of natural numbers $\mathbb{N}$.
 
\textbf{Arbitrary Relations}:
Relations need not be just binary. For example, our normal interpretation of the $+$ symbol (along with equality) specifies a relation - we might think of the ternary relation $+abc$ as expressing that $a + b = c$. The extension of this relation is then a set of ordered triples $\{(a, b, c) | a + b = c\}$. 

\textbf{Graphs}:
A (directed) \emph{graph} is a structure $G = (V,E)$ suitable for interpreting a dyadic predicate symbol $L$.  $V$ is a set of ``vertices'' or ``nodes'', and $E$ is a binary relation (the edge relation) on $V$ (that is, $E \subseteq V \times V$). We will often write $U^G$ for $V$ and $L^G$ for $E$ when we use $G$ to interpret schemata involving only the dyadic predicate symbol $L$.

\textbf{Neighborhood}: The neighborhood of a node $n$ of a graph $G=(V,E)$ is the set of all nodes which are adjacent to $n$ in $G$ (that is, $\{n'\in V | (n,n')\in E \}$).

\textbf{Degree}: The degree of a node $n$ is the size of $n$'s neighborhood. 


\textbf{Reflexive}: A relation $R$ is said to be \emph{reflexive} iff $(\forall x)Rxx$. 

If we draw an arrow diagram of such a relation, each vertex would have a ``self loop.'' If we visualize such a relation as a bit matrix, the main diagonal is all ``1s.'' 

\textbf{Irreflexive}: A relation $R$ is said to be \emph{irreflexive} iff $(\forall x)\lnot Rxx$.

If we draw an arrow diagram of such a relation, this specifies that there are no ``self loops.'' 

\textbf{Symmetric}: A relation $R$ is said to be \emph{symmetric} iff $(\forall x)(\forall y)Rxy \supset Ryx$.

This means that there are no one-way relations between elements. Specifically, either $(Rxy \land Ryx) \oplus (\lnot Rxy \land \lnot Ryx)$. When drawing an arrow diagram of a symmetric relation, we often omit the arrowheads on our edges to indicate that the relation ``goes both ways.'' 

\textbf{Antisymmetric}: A relation $R$ is said to be \emph{antisymmetric} iff $(\forall x)(\forall y)Rxy \supset \lnot Ryx$.

This means that there are no two-way relationships between elements. 

\textbf{Transitive}: A relation $R$ is said to be \emph{transitive} iff $(\forall x)(\forall y)(\forall z)((Rxy \land Ryz) \supset Rxz)$ or, equivalently (why?) $(\forall x)(\forall y)(\forall z)(Rxy \supset (Ryz \supset Rxy))$.

\textbf{Comparable}: A relation is $R$ said to be \emph{comparable} iff $(\forall x)(\forall y)(x \neq y \supset (Rxy \vee Ryx))$. 

In words: if $x$ and $y$ are unequal, then they have to be related in some way (so every pair of nodes is related somehow). 

\textbf{Simple Graph}: A relation is said to be a \emph{Simple Graph} iff the relation is both irreflexive and symmetric. We abbreviate this $\sg$. 

\textbf{Tournament}: A relation is said to be a \emph{Tournament} iff the relation is both antisymmetric and comparable. Think of the nodes in the graph as being teams, and the edge relation as being ``won against'' in a round robin - each team plays each other (so they all have at least one edge between) and only one team wins each game (so if $a$ beats $b$, then $b$ does not beat $a$). We abbreviate this $\tour$. 

\textbf{Strict Linear Order}: A relation is said to define a \emph{Strict Linear Order} iff the relation is antisymmetric, comparable, and transitive (thus, SLOs are a subclass of Tournaments). The extra criterion of transitivity requires that if $a$ ``beats'' $b$, then $a$ also ``beats'' every team which $b$ beats. In this way, the ``1st place team'' beats everyone, 2nd place beats everyone except first, \emph{etc.}, and so there is a strict ordering on the elements of our graph. 

\textbf{Equality}: Equality is a special type of relation in that its interpretation is uniform across structures. For every structure $A$, equality holds just of the pairs $\{(a, a) | a \in U^A\}$.

\textbf{Counting Quantifiers}: In MQT, the \emph{Small Model Property} meant we had no way to distinguish between, for example, the two structures 
\[
    A: U^A = \{1, 2\}, P^A = \{1, 2\}
\]
and
\[
    B: U^B = \{1\}, P^B = \{1\}
\]
because $A, B$ realised the same types, \emph{even though} a different number of elements of $A$ are in $P$ than elements of $B$ are. With $PQT$, we can of course distinguish between these two structures by \emph{using equality to count}. 

We defined sentences $\Delta_n:= (\exists x_1)...(\exists x_n)\bigwedge_{1 \leq i < j \leq n}x_i \neq x_j$ which indicates that elements $x_1...x_n$ are all pairwise distinct. Using this, we defined
\[
    (\exists^{k \leq}x)P(x)
\]
which says there are at least $k$ elements such that $P$
\[
    (\exists^{\leq k}x)P(x)
\]
which says that there are at most $k$ elements such that $P$, and
\[
    (\exists^{=k})P(x)
\]
which says that there are exactly $k$ elements such that $P$. 

\textbf{k-Regular}: We say that a graph $G=(V,E)$ is $k$-regular iff every node $n \in V$ has degree $k$. Finite $2$-regular simple graphs are composed of a collection of disjoint cycles, and $1$-regular simple graphs are composed of a collection of disjoint ``isolated edges.'' 

\textbf{Counting Graphs}: We defined
\[
    \modn{S}{n} := \{A | A \models S \mbox{ and }U^A = \{1,...,n\}\}
\]
That is, $\modn{S}{n}$ is the collection of all structures satisfying a sentence $S$ which have universe $\{1,...,n\}$. 

\textbf{Functions as Relations}:
Functions are a special subclass of relations. Specifically, they are relations which satisfy the two properties 
\[
    \tot: (\forall x)(\exists y)Rxy
\]
and
\[
    \sv: (\forall xyz)((Rxy \land Rxz) \supset y = z).
\]

The first says that the relation is \emph{total} - for everything in the domain, there is something to which it maps. The second says that the relation is single valued - every element of the domain maps to at most one element. The conjunction $\tot \land \sv$ implies that every element of the domain maps to exactly one element of the codomain. 
Strictly, $\tot$ is not required for a relation to be a function (the class of relations axiomatized by only $\sv$ are called the \emph{partial functions}). In this class, however, we restrict our attention to total functions, and will always mean ``total function'' whenever we say ``function.'' 

\textbf{Injectivity}: A function is injective (also sometimes called ``one-to-one'') iff the following holds
\[
    \inj: (\forall xyz)((Rxz \land Ryz) \supset x = y) 
\]
which says that no two elements of the domain map to the same element of the codomain. %Note that this is the converse of $SV$. 

If there is an injection from $S$ to $S'$ then $|S| \leq |S'|$ (prove this for finite sets). 

\textbf{Surjectivity}: A function is surjective (also sometimes called ``onto'') iff the following holds:
\[
    \sur: (\forall x)(\exists y)Ryx
\]
which says that every element of the codomain is the image of some element of the domain. %Note that this is the converse of $Tot$. 

If there is a surjection from $S$ to $S'$, we know that $|S| \geq |S'|$ (prove this for finite sets). 

\textbf{Functions on a Single Set}:
We often consider functions with the same domain and codomain (say $S$). In this case, we say that the function is on $S$. 
We proved that, for finite sets $S$, a function on $S$ is an injection iff it is a surjection. 
In contrast, we showed that there are injections on an infinite $S$ which are not surjections, and similarly that there are surjections on an infinite $S$ which are not injections. For example
\[
    f: \mathbb{N} \rightarrow \mathbb{N}
\]
\[
    n \mapsto n + 1
\]
is an injection but not a surjection (0 is not mapped to), and 
\[
    h: \mathbb{N} \rightarrow \mathbb{N}
\]
\[
    n \mapsto \lfloor n/2\rfloor
\]
is a surjection but not an injection (for all $k$, $2k$ and $2k + 1$ both map to $k$).

A set $S$ is called \emph{Dedekind infinite} if and only if
there is an injection on $S$ that is \emph{not} a surjection.
%We remarked that \emph{the capability of $S$ having an injection on it which is not surjective} was Dedekind's characterization of what it meant for a set $S$ to be infinite. 

\textbf{Binary Functions}: Just as the \emph{graph} of a unary function $f$ is the binary relation $\{(a,b)\mid f(a)=b\}$, the \emph{graph} of a binary function $g$ is the ternary relation $\{(a,b,c)\mid g(a,b)=c\}$. The following schema is satisfied by a structure $A$ if and only if $R^A$ is the graph of a total binary function on $U^A$.
\[
    \bfun:= (\forall x)(\forall y)(\exists z)(\forall w)(Rxyw \equiv w = z).
\]
%which says that every pair of $x,y$ maps to a unique $z$. Finding a sentence expressing totality for binary functions is left as an exercise (build the schema by analogy to $Tot$ for unary functions). 

\textbf{Binary Injection}: Similarly, the following schema expresses that a binary function is injective. 
\[
    \binj:= (\forall vwxyz)((Rvwz \land Rxyz) \supset (v = x \land w = y))
\]
%which says that each $z$ in the codomain is mapped to by exactly one pair $x, y$ from the domain. 
\begin{aside}
Find a schema that expresses surjectivity for binary functions.% is left as an exercise (build the schema by analogy to $Surj$ for unary functions). 
\end{aside}
\textbf{Counting Relations}: We have the following:
\[
    |\modn{\fun}{n}| = n^n,
\]
that is, there are $n^n$ functions on a set of size $n$. To see this, note that each for each of the $n$ elements we must choose one of $n$ possible images. In other words, to determine a function on $[n]$, we must make $n$ $n$-ary choices.%there are $n$ has $n$ choices for which element it maps to.
\[
    |\modn{\fun\land\inj}{n}| = n!
\]
To see this, note that a function on a set $S$ which is injective maps each of the $n$ elements to a unique element. So there are $n$ choices for the first element, $n-1$ for the second, \textellipsis, and $1$ for the last. 
\[
    |\modn{\asy}{n}| = 3^{\binom{n}{2}}
\]
To see this, note that for each pair of elements $\{a, b\}$ there are three mutually exclusive possibilities - no edge, edge from $a$ to $b$, or edge from $b$ to $a$. 
\[
    |\modn{\tour}{n}| = 2^{\binom{n}{2}}
\]
To see this, note that for each pair of elements $\{a,b\}$ there must be exactly one of the directed edges  from $a$ to $b$ or from $b$ to $a$. 
\[
    |\modn{\slo}{n}| = n!
\]
To see this, note that there is a natural one-to-one correspondence between linear orderings of $[n]$ and injections on $[n]$ -- map each $i\in[n]$ to the $i$-th member of the order; thus, there are the same number of strict linear orders of $[n]$ as there are injections on $[n]$.% $n!$ ways of ordering $n$ elements. 
\[
    |\modn{\twor}{3}| = 1
\]
To see this, note that a 2-regular graph of size 3 must be a ``triangle'', and there is only one way of making a triangle with three elements. 
\[
    |\modn{\twor}{4}| = 3
\]
To see this, note that a 2-regular graph of size 4 must be a ``square''. Fix one element $a$ of the square, and pick the single node from the remaining three to which it is not adjacent (three choices). This uniquely determines the square. 

%Equivalently, consider labelling the four vertices of the square. This can be done in $4!$ ways. However, there are $4$ ways to ``rotate the square'' which result in equivalent graphs. Similarly, the square can be ``flipped'' (flipped vs non-flipped: 2 choices) across a line of symmetry to give an equivalent graph. So there are $\frac{4!}{4\cdot 2} = 24/8 = 3$ such graphs.

In general, there are $\frac{n!}{n\cdot 2} = \frac{(n-1)!}{2}$ $n$-cycle graphs. 
\[
    |\modn{\twor}{6}| = 70
\] 
Any 2-regular graph of size six must be composed of two ``triangles'' or one ``hexagon.'' For the triangles, there are $\binom{6}{3}/2 = 10$ ways to split the six up into two groups of three (we divide by two because, say, picking elements $\{1, 2, 3\}$ for one triangle is the same as picking $\{4, 5, 6\}$ for the other). Each such division corresponds to exactly one graph, so there are 10 graphs which have two disjoint triangles. For the hexagon, there are (reasoning as above) $\frac{6!}{6 \cdot 2} = 60$ ways to label the hexagon, giving $10 + 60 = 70$ total possible graphs. 

\textbf{Spectrum}: We defined the \emph{spectrum} of a schema $S$ to be the set of all integers $n$ for which there is a model of $S$ of size $n$. We determined that for each positive integer $n$, there is a schema $\mathsf{S}_n$ such that $\spec{\mathsf{S}_n} = \{n\}$. The schema $\mathsf{S}_n$ specifies that there are exactly $n$ elements in the universe.

By making use of the schemata $\mathsf{S}_n$, we constructed, for each finite set $F$ and for each cofinite set $C$, schemata $S$ and $T$ with $\spec{S}=F$ and $\spec{T}=C$. (Recall that a set of positive integers $X$ is \emph{cofinite} if and only if $\mathbb{Z}^+ - X = \overline{X}$ is finite)

%We realized that any finite set is the spectrum of some schema. In general, for any finite set $K = \{k_1,...,k_n\}$, $S := S_{k_1} \vee ... \vee S_{k_n}$ has spectrum $K$. 

%We called a set $cofinite$ iff its complement is finite (that is, $X$ is cofinite iff $\mathbb{N} - X = \overline{X}$ is finite). Because we can define ever finite set, it followed that we could define every cofinite set. For example, if we wanted a spectrum of $\mathbb{N} - \{1, 5, 7, 11, 23\}$, the schema $S := \lnot(S_1 \vee S_5 \vee S_7 \vee S_{11} \vee S_{23})$ suffices. 

We noted that, in general, $\spec{\lnot S} \neq \overline{\spec{S}}$. For example, let $S := \sg \land \oner$, so that $Spec(S) = \{2i | i \in \mathbb{Z}^+\}$. Then $\lnot S$ is equivalent to $\lnot \sg \vee \lnot \oner$. Therefore, $\spec{\lnot S} = \mathbb{Z}^+$. 


\end{mdframed}



\newpage
\begin{mdframed}[linewidth=1]
\section*{Problems}
\begin{enumerate}
    \item We say that a graph $G \in \modn{S}{n}$ is \emph{size maximal} iff it has the most edges possible given that it satisfies $S$ and has $n$ nodes. Give an example of a size-maximal acyclic member of $\modn{\sg}{3}$. 

    \item How many size-maximal acyclic members of $\modn{\sg}{3}$ are there?

    \item How many size-maximal acyclic members of $\modn{\sg}{4}$ are there?

    \item Prove that if every node of a finite graph is of degree at least 2, then the graph contains a cycle. 
\iffalse
\item How many size-maximal members of $\modn{\sg}{6}$ have no 3-cycles? 

    \item Prove the \textbf{Handshake Lemma}: For every graph $G=(V,E)$, $\sum_{v \in V}\deg{v} = 2|E|$. 

    \item Prove that in any graph the number of vertices of odd degree is even.
\fi

    \item Let $S$ be the conjunction of 
   \[
       \slo
   \]
   \[
       (\forall x)(\forall y)((Lxy \land \lnot (\exists z)(Lxz \land Lzy)) \supset Fx \oplus Fy)
   \]
   \[
       (\forall x)(\lnot (\exists y)Lyx\supset Fx)
   \]
   \[
       (\forall x)(\lnot (\exists y)Lxy\supset \lnot Fx)
   \]
   What is $\spec{S}$?

   \item Give a schema $S$ such that $\spec{S} = \{2i + 1 | i \in \mathbb{N}\}$, that is, the set of odd positive integers. 

   \item What is $\spec{\slo}$?

   \item Let $S$ be the conjunction of 
   \[
       \slo
   \]
   \[
       (\forall x) (\exists y) Lxy
   \]
   \[
       (\forall x) (\exists y) Lyx
   \]
   \[
       (\forall x)(\forall y)(Lxy \supset (\exists z)(Lxz \land Lzy))
   \]
   What is $\spec{S}$?

   \item Let $S$ be the conjunction of
   \[
       (\forall x)((\forall y) \lnot Lxy \vee (\exists y)(\exists z)(y \neq z \land (\forall u)(Lxu \equiv (u = y \vee u = z))))
   \]
   \[
       (\forall x)(\forall y)(Lxy \supset \lnot Lyx)
   \]
   \[
       (\exists x)((\forall y) \lnot Lyx \land (\forall y)(y \neq x \supset (\exists u)(\forall w)(Lwy \equiv u = w)))
   \]
   What is $\spec{S}$?

\end{enumerate}
\end{mdframed}

\newpage
\begin{mdframed}[linewidth=1]
\section*{Solutions}
\begin{enumerate}
    \item $U^A = \{1, 2, 3\}$, $R^A = \{(1, 2), (2, 1), (2, 3), (3, 2)\}$ is one such example. 

    \item $\binom{3}{2} = 3!/2 = 3$. Note that all size-maximal acyclic members of $mod(SG, 3)$ have two undirected edges (if they had a third, there would have to be a cycle). So to specify such a graph, we simply have to pick which two nodes are not connected. Equivalently, you might note that all such graphs look like a single chain (or, a linear order in which direction doesn't matter). There are $3!$ orders, and we divide by $2$ because reading from right-to-left or left-to-right doesn't matter (we have a simple graph, so the edges are undirected). 

    \item $4!/2 + 4 = 16$. As above, there are $4!/2$ ways to order the elements into a single chain. However, unlike the case above, there are graphs which aren't just a single chain. Picking one element to be the ``parent'' and placing undirected edges from it to every other element (draw this!) gives a graph which is also size maximal acyclic, but not a chain. There are 4 ways to pick such a parent. So the total is $4!/2 + 4 = 16$. 

    In general, acyclic size maximal members of $\modn{\sg}{n}$ are \emph{trees} of size $n$ (a ``tree'' is a size-maximal acyclic graph, or equivalently a size-minimal connected graph).
    
    \item Consider a maximal length path (an acyclic sequence of edges) in the graph. As the terminal node of that path has degree at least 2, there must be an edge coming from it which is not on our path. As our path was maximal in length, that edge must go back to some previously visited node along our path.   
\iffalse 

    \item There are 10. Complete bipartite graphs which have equal numbers of red/blue elements are the only such graphs (convince yourself of this by drawing out cases/examples). There are $\binom{6}{3}/2 = 10$ ways to split up the nodes into red/blue.

    \item Note that adding any edge to a graph increases the sum of degrees by 2, as it increases the degree of each of its two incident nodes by one. 

    \item Suppose there were an odd number of such vertices. Then the sum of degrees would be odd. This contradicts the handshake lemma.

\fi    
    \item $\{2i | i \in \mathbb{Z}^+\}$. 

   To answer this question, we interpret $S$ piece by piece. The first conjunct is $\slo$, so we treat this like a linear order from now on (in particular, we can interpret $L$ as meaning ``less than''). 

   The second conjunct says that ``for all $x$ and $y$, if there isn't a $z$ between them, then exactly one of $x$ and $y$ are in $F$''. Interpreting $Fx$ as meaning ``$x$ is colored red'' and $\lnot Fx$ as meaning ``$x$ is colored blue'', we interpreted this sentence as saying that the coloring of nodes in our order switches from red to blue every element. 

   The third conjunct says that if there is no element less than $x$ (so $x$ is the minimal element), then $x$ is red.

   The final conjunct says that if there is no element greater than $x$ (so $x$ is the maximal element), then $x$ is blue. 

   Putting this all together, we have a strict linear ordering whose first node is red, node color switched every adjacent node, and the last node is blue. It follows that we have an even number of nodes, giving us our answer.  

    \item Let $S$ be the conjunction of 
    \[
       \slo
   \]
   \[
       (\forall x)(\forall y)((Lxy \land \lnot (\exists z)(Lxz \land Lzy)) \supset Fx \oplus Fy)
   \]
   \[
       (\forall x)(\lnot (\exists y)Lyx\supset Fx)
   \]
   \[
       (\forall x)(\lnot (\exists y)Lxy\supset  Fx)
   \]   
   The only difference from before is in the last conjunct; $\lnot Fx$ was replaced with $Fx$. $S$ now says we have a linear order, flipping colors, both ending and starting with red. So only odd-sized orders work now.

    \item $\spec{\slo} = \mathbb{Z}^+$, as there are linear orders of every finite size. 

    \item $\spec{S} = \emptyset$. The second conjunct says that there is no maximum element, the third says there is no minimum, and the final says that each pair of elements has a third element in between the two. Any of these conjuncts, together with $\slo$, requires that all models be infinite. In particular, there is no $n \in \mathbb{Z}^+$ with a model of size $n$, and so, by the definition of spectrum, $\spec{S} = \emptyset$. 

   Note, of course, that this does not mean that there are no models of $S$ - if we interpret $L$ as $<$, then $\mathbb{Q}$, the rational numbers together with its usual strict linear ordering. is a model of $S$. It only means that there are no finite models. 

    \item $\spec{S} = \{2i + 1 | i \in \mathbb{N}\}$. 

   The first schema says that, for all $x$, either $x$ is related to no element, or $x$ is related to exactly two elements. 

   The second schema says that $L$ is asymmetric. 

   The third schema says that there is an element which is not related to by anything, and that all other elements are related to by exactly one element.

   It follows that models of $S$ look like rooted, directed trees in which every node has either two children or no children. Since each parent corresponds to an even number of children and the only non-child node is the root node, there must be an odd number of nodes. Moreover, we may construct, for each $n$, a structure $T_n$ satisfying $S$ with $2n+1$ nodes. $T_0$ is a single isolated root node. Given $T_n$, we construct $T_{n+1}$ by adding two nodes as children of the leftmost leaf of $T_n$. The result now follows. 
   
\end{enumerate}
\end{mdframed}