We considered the problem of establishing that a schema $S$ is not implied by a set of schemata $X$, or equivalently, that the set of schemata $X\cup\{\neg S\}$ is not satisfiable. As we noted last time, there is no uniform approach to this problem, that is, the collection of satisfiable schemata is \emph{not} semi-decidable.

Let $X$ be the conjunction of the following schemata.
\begin{itemize}
\item 
$(\forall x)(\forall y)(\forall z)((Lxy \wedge Lyz) \supset Lxz)$
\item
$(\forall x)(\forall y)(x\neq y\supset(Lxy \vee Lyx))$
\item
$(\forall x) \neg Lxx$
\item 
$(\forall x)(\exists y)(Lxy\wedge (\forall z)\neg (Lxz\wedge Lzy))$
\item 
$(\forall x)(\exists y)(Lyx\wedge (\forall z)\neg (Lyz\wedge Lzx))$
\item
$(\forall x)(\exists y)(Lyx\wedge Fy)$
\item
$(\forall x)(\exists y)(Lxy\wedge Fy)$
\item
$(\forall x)(\forall y)((Fx\wedge Fy\wedge Lxy)\supset (\exists z)(Fz\wedge Lxz\wedge Lzy))$
\end{itemize}

We showed that $X\not\models(\forall x)Lxx$, that is, we showed $X$ is satisfiable by constructing a structure $A$ with $A\models X$. The structure $A$ is defined as follows. Recall that $\mathbb{Z}$ is the set of integers and $\mathbb{Q}^+$ is the set of positive rational numbers.
\begin{itemize}
\item
$U^A= \mathbb{Q}^+\times\mathbb{Z}=\{\op{r}{i}\mid r\in\mathbb{Q}^+\mbox{ and}\ i\in \mathbb{Z}\}$ (the cartesian product of $\mathbb{Q}^+$ and $\mathbb{Z}$).
\item
$L^A=\{\op{\op{r}{i}}{\op{s}{j}}\mid r<s\}\cup\{\op{\op{r}{i}}{\op{s}{j}}\mid r=s\mbox{ and }i<j\}$.
\end{itemize}

We gave another example of demonstrating satisfiability, this time for an infinite collection of schemata. Let $S$ be the conjunction of the following schemata.
\begin{itemize}
\item 
$(\forall x)(\forall y)(\forall z)((Lxy \wedge Lyz) \supset Lxz)$
\item
$(\forall x)(\forall y)(x\neq y\supset(Lxy \vee Lyx))$
\item
$(\forall x) \neg Lxx$
\item 
$(\forall x)((\exists y)Lxy\supset(\exists y)(Lxy\wedge (\forall z)\neg (Lxz\wedge Lzy)))$
\item 
$(\forall x)((\exists y)Lyx\supset(\exists y)(Lyx\wedge (\forall z)\neg (Lyz\wedge Lzx)))$
\item
$\neg(\forall x)(\exists y)Lyx$
\item
$\neg(\forall x)(\exists y)Lxy$
\end{itemize}
For each $n\geq 2$, let $R^n$ be the schema, 
\[
(\exists x_1)\ldots(\exists x_n)\bigwedge_{1\leq i< j\leq n}Lx_ix_j.
\]
Finally, let $X=\{S\}\cup\{R^n\mid n\geq 2\}$. We gave two proofs that $X$ is satisfiable. The first appealed to the 
\begin{theorem}[Compactness Theorem]
Let $\Sigma$ be a set of schemata of polyadic quantification theory. If every finite $\Delta\subseteq\Sigma$ is satisfiable, then $\Sigma$ is satisfiable.
\end{theorem}

\emph{First Proof}:
Observe that for every $n\geq 2$, $\{S\}\cup \{R^m\mid m\leq n\}$ is satisfied by a linear order of length $n$. Hence, by the Compactness Theorem, $X$ is satisfiable. \qed

\emph{Second Proof}: Define the structure $B$ as follows.
\begin{itemize}
\item
$U^B=  \mathbb{Z}$.
\item
$L^B=\{\op{i}{j}\mid (0\leq i\mbox{ and } j<0)\mbox{ or }(i<j \mbox{ and }  (0\leq i,j\mbox{ or } i,j<0))\}$.
\end{itemize}
Observe that $B\models X$. \qed
\end{document}
\begin{itemize}
\item $(\forall x)(Fx\supset(\exists y)(\neg Fy\wedge(\forall z)(Lxz\equiv y=z)))$
\end{itemize}
For each $n\geq 2$, let $R_n$ be the schema
\[
(\forall y)(\neg Fy\supset(\exists x_1)\ldots(\exists x_n)\bigwedge_{1\leq i<j\leq n}(x_i\neq x_j\wedge Fx_i\wedge Lx_iy));
\]
and for each $n\geq 2$, let $T_n$ be the schema
\[
(\exists x_1)\ldots(\exists x_n)\bigwedge_{1\leq i<j\leq n}(x_i\neq x_j\wedge \neg Fx_i).
\]
Let $X=\{S,R_n,T_n\mid n\geq 2\}$. We showed that $X$ is satisfiable by constructing a structure $B$ with $B\models X$.
\begin{itemize}
\item
$U^B= \mathbb{Z}^+$.
\item
$F^B=\{2i\mid i\in\mathbb{Z}^+\}$.
\item
$L^B = \{\op{2^i\cdot j}{j}\mid i\in\mathbb{Z}^+\mbox{ and } j\not\in F^B\}$.
\end{itemize}
