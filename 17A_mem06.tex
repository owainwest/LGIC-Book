\subsection{Monadic Quantification Theory}
\subsubsection{Sub-sentential logical structure: monadic predicates}
 
We then initiated our study of monadic quantification theory. Statements have significant logical form beyond the structure that can be
exhibited in terms of truth-functional compounding. For example, the conjunction of the first two statements below implies, but does not truth-functionally imply, the third.
%\begin{dispar}\label{lassie-ex}
\begin{itemize}
\item All collies are mortal. 
\item Lassie is a collie.
\item Lassie is mortal.
\end{itemize}
%\end{dispar}
In order to analyze this example, we considered the following statements.
\begin{itemize}
\item Lassie is a collie.
\item Scout is a collie.
\item Rin-Tin-Tin is a collie.
\end{itemize}
These statements share the {\em monadic predicate} ``$\bigcirc$\ is a collie.''
Monadic predicates, unlike statements, are not true or false; rather, they are
{\em true of} some objects and {\em false of} other objects.
For example, ``$\bigcirc$ is a prime number'' is true of 2,3,5 and 7, and false of all even
numbers greater than 2.
\subsubsection{The extension of a monadic predicate}

The {\em extension} of a monadic predicate is the collection of objects of
which it is true. The extension of the monadic predicate ``$\bigcirc$ is an even number'' is the
set $\{2,4,6,\ldots \}.$
The extension of the monadic predicate ``$\bigcirc$ is an even prime number''
is the set
 $\{2 \}.$
The extension of the monadic predicate ``$\bigcirc$ is an even prime number
greater than 2''
is the empty set. 

Distinct monadic predicates may have the same extension. For example, the
extension of the predicate ``$\bigcirc$ is a warm--blooded reptile'' is also
the empty set as is the extension of the predicate
``$\bigcirc$ is a collie weighing more than 300 kilograms.''
We say that monadic predicates with the same extension are {\em coextensive}.

We will focus on statements whose truth depends only on the extensions of the
monadic predicates which occur in them. We call such sentential contexts 
in which interchange of coextensive predicates preserves truth--value {\em
extensional}. Our focus on extensional contexts is the natural continuation of
our earlier focus on truth--functional contexts.
\subsubsection{Open sentences and the use of variables}

Consider again the argument above.
Intuitively, the validity of this argument does not depend on the particular
name ``Lassie'' being used; it would be equally valid with any name in place of
``Lassie.'' 
This generality may be brought out by the use of variables in place of
particular names. 
We will form new expressions called {\em open sentences} by putting variables
``$x,y,z,\dots $'' for the placeholders in monadic predicates.
Open sentences are not statements. They are true or false with respect to
assignments of values to the variables they contain. For example, the open
sentence ``$x$ is an even number'' is true with respect to the assignment of 16
to 
``$x$'' and false with respect to the assignment of 17  to ``$x$'' and false
with respect to the assignment of Lassie to ``$x$.''
\subsubsection{Truth-functional compounding of open sentences}

We may form compounds of open sentences using truth--functional connectives.
For example, the following open sentences are truth--functionally complex.
\begin{itemize}
\item If $x$ is divisible by six, then $x$ is divisible by three.
\item $x$ is a collie and it is not the case that $x$ weighs more than 300 kg. 
\end{itemize}
We may use our prior understanding of the truth--functional connectives to
determine the truth--values of such open sentences with respect to particular
assignments of values to their variables.  
\subsubsection{Existential Quantification}

We proceeded to introduce the existential quantifier. Consider the statement,
``there is an even prime number.''
We render this statement as the application of the existential quantifier to the open sentence,
\begin{itemize}
\item $x$ is an even number $\wedge$ $x$ is a prime number, thus
\item $(\exists x)$($x$ is an even number $\wedge$ $x$ is a prime number).
\end{itemize}
This last sentence is true just in case there is an assignment of some object to
the variable $x$ with respect to which the preceding open sentence is true.
\subsubsection{Free and bound occurrences of variables}

 Consider again the example the example above. 
\begin{itemize}
\item $x$ is an even number $\wedge$ $x$ is a prime number
\item $(\exists x)$($x$ is an even number $\wedge$ $x$ is a prime number)
\end{itemize}
As noted, the first of these sentences is not simply true or false, it is true or false with respect to an assignment to the variable ``$x$''; we say in this instance that the occurrences of the variable ``$x$'' are \emph{free} in this sentence. On the other hand, the occurrences of the variable $x$ are \emph{bound} by the existential quantifier in the second sentence; this sentence is true or false independent of any assignment to the variable $x$. Note that a variable may have both free and bound occurrences within a single sentence:
\begin{itemize}
\item $(\exists x)$($x$ is an even number) $\wedge$ ($x$ is a prime number); \end{itemize}
and may have occurrences bound by distinct quantifiers: 
\begin{itemize}
\item $(\exists x)$($x$ is an even number) $\wedge$ $(\exists x)$($x$ is a prime number).
\end{itemize}
\subsubsection{Universal Quantification}

Next we consider the use of the universal quantifier. We can render the statement
\begin{itemize}
\item all numbers are even or odd
\end{itemize}
as
\begin{itemize}
\item $(\forall x)$ [($x$ is an even number) or ($x$ is an odd number)].
\end{itemize}
The last statement is true, just in case whatever integer is assigned to the variable $x$ satisfies the open statement within the square brackets. Here we see the contextual determination of a \emph{universe of discourse} -- when we say ``all numbers'' in this context, we intend that the variable of quantification range over all integers and not, for example, all complex numbers.
\subsubsection{Monadic Schemata}

As we did in the case of truth-functional logic, we will introduce a schematic language for monadic quantificational logic. We specify the following categories of monadic schemata.
\begin{itemize}
\item 
A {\em one variable open schema} is a truth functional compound of expressions
such as \\
$Fx , Gx , Hx , \ldots .$
\item
A {\em simple monadic schema} is the existential or universal quantification of
a one variable open schema with variable of quantification $x.$
\item
A {\em pure monadic schema} is a truth functional compound of simple monadic
schemata. 
\end{itemize}
\subsubsection{Structures as interpretations of monadic schemata}

We introduce \emph{structures} as interpretations of monadic schemata. These play the role that truth-assignments played in the context of truth-functional logic.
In order to specify a structure $A$ for a schema $S$ we need to
\begin{itemize}
\item specify a nonempty set $U^A,$ the universe of $A$;
\item specify sets $F^A, G^A, \ldots $ each of which is a subset of $U^A$ as
the extensions of the monadic predicate letters which occur in $S$;
\item specify an element $a \in U^A$ to assign to the variable $x,$ if $x$
occurs free in $S.$ 
\end{itemize}
When the variable $x$ has no free occurrences in the schema $S$, we write $A \models S$ as shorthand for ``the schema $S$ is true in the
structure $A,$'' alternatively ``the structure $A$ satisfies the schema $S$.'' Otherwise, we write $A\models S[a]$ as shorthand for ``the structure $A$ satisfies the schema $S$ relative to the assignment of $a$ to the variable $x$.''  
\subsubsection{Validity, satisfiability, implication, and equivalence}

We extend the notions of validity, satisfiability, implication, and equivalence to monadic quantificational schemata.
\begin{itemize}
\item
A monadic schema $S$ is {\em valid} if and only if for every structure $A, A
\models S.$
\item
A monadic schema $S$ is {\em satisiable} if and only if for some structure $A,
A \models S.$
\item
A monadic schema $S$ {\em implies} a monadic schema $T$ if and only if for
every structure $A,$ if $A \models S,$ then $A \models T.$
\item
Monadic schemata $S$ and $T$ are equivalent if and only if $S$ implies $T$, and $T$ implies $S$. 
\end{itemize}
\subsubsection{Counting the number of structures with fixed universe of discourse that satisfy a schema}

We discussed how to count the number of structures with a fixed universe of discourse that satisfy a given schema. We asked, how many structures with universe of discourse $U=\{1,2,3,4,5,6\}$ interpreting the monadic predicate letters $F$ and $G$ satisfy the schema 
\[S:\ \ \ (\forall x)(Fx\supset Gx).\] 
We observed that a structure $A$ satisfies $S$ if and only if $F^A\subseteq G^A$. So we need to determine the number, call it $n$, of pairs of subsets $Y,Z$ of $U$ with $Y\subseteq Z$. By using what we learned earlier about binomial coefficients, we see that 
\[n=\sum_{i=0}^{i=6}\binom{6}{i}2^i = \sum_{i=0}^{i=6}\binom{6}{i}2^i\cdot 1^{6-i}=(2+1)^6 = 3^6.\]
The next to last equality is justified by the celebrated \emph{Binomial Theorem}. For those of us with no taste for binomial coefficients, we will discuss a much simpler and direct combinatorial argument for the conclusion that $n= 3^6$.
\subsubsection{Element Types}

Consider the following four one variable open schemata; we will call them (element) types.
\begin{itemize}
\item $T_1(x): Fx\wedge Gx$
\item $T_2(x): Fx\wedge \neg Gx$
\item $T_3(x): \neg Fx\wedge Gx$
\item $T_4(x): \neg Fx\wedge \neg Gx$
\end{itemize}
Note that a structure $A$ satisfies the schema $S$ if and only if it contains no element satisfying the type $T_2$. Since a structure is determined by the type of each of its elements, there are as many structures with universe $U$ satisfying $S$ as there are ways of sorting the members of $U$ into the three remaining types. For each of the six members of $U$, there are three types into which it could be sorted, so by the product rule, the number of structures satisfying $S$ is $3^6$.
\subsubsection{Counting counterexamples to an alleged implication}
If $R$ and $R^*$ are monadic schemata we say that a structure $A$ is a \emph{counterexample} to the claim that $R$ implies $R^*$ if and only if $A\models R$ and $A\not\models R^*$. We continued with the preceding example and counted the number of counterexamples to the claim that the schema $S$ implies the schema
\[T:\ \ \ \ (\forall x)(Gx\supset Fx).\]
Again, we suppose that our structures have universe of discourse $U$ and interpret exactly the monadic predicate letters $F$ and $G$. If a structure $A$ satisfies both $S$ and $T$, then $F^A=G^A$. Hence, of the $3^6$ structures satisfying $S$, the number that also satisfy $T$ is $2^6$, that is, the number of subsets of $U$, assigned within a single structure to both $F$ and $G$. So the number of counterexamples to the claim that $S$ implies $T$ is $3^6 - 2^6$.  
