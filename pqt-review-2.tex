\subsection{Review}
\begin{mdframed}[linewidth=1]
\section*{Concept Review}
test

\textbf{Equivalence Relations}: An equivalence relation is a relation which satisfies
\[
    Ref: (\forall x)Lxx
\]
\[
    Sym: (\forall xy)Lxy \supset Lyx
\]
\[
    Trans: (\forall xyz)((Lxy \land Lyz) \supset Lxz)
\]
We called $\hat{a} = \{a' \in U^A | Laa'\}$ the \emph{equivalence class of $a$}; it is the set of all $a' \in U^A$ which are equivalent to $a$ under our equivalence relation. 

We proved that these properties ensured that the equivalence classes of such a relation partition the universe, that is that every element of the universe is in some equivalence class (this follows by reflexivity) and that the different equivalence classes are all disjoint (this follows by symmetry and transitivity). 

For all $k, l \in \mathbb{N}$ we can define the spectrum $\{ki + l | i \in \mathbb{N}\}$ (the set of all integers equivalent to $l$ mod $k$) by specifying that there are exactly $l$ elements in an equivalence class of size $l$, and that all other elements are in size-$k$ equivalence classes. 

\end{mdframed}



\newpage
\begin{mdframed}[linewidth=1]
\section*{Problems}
\begin{enumerate}
    \item 


\end{enumerate}
\end{mdframed}

\newpage
\begin{mdframed}[linewidth=1]
\section*{Solutions}
\begin{enumerate}
    \item 
\end{enumerate}
\end{mdframed}