\subsection*{Counting Graphs}
As before, we'll count the number of finite structures with universe of discourse $\{1,\ldots,n\}$ that satisfy various conditions. We already know that there are $2^{n^2}$ graphs and $2^{\binom{n}{2}}$ simple graphs with universe of discourse $\{1,\ldots,n\}$. It is simple to show that
\begin{itemize}
\item
$\card{\modn{\fun}{n}}= n^n$;
\item
$\card{\modn{(\fun\wedge\inj)}{n}}= n!$;
\item
$\card{\modn{\asy}{n}}= 3^{\binom{n}{2}}$;
\item
$\card{\modn{\tour}{n}}= 2^{\binom{n}{2}}$;
\item
$\card{\modn{\slo}{n}}= n!$;
\item
$\card{\modn{\bfun}{n}}= n^{(n^2)}$.
\end{itemize}

\begin{aside}
    Prove each of the above assertions by counting them yourself. 
\end{aside}

Since you're probably a pro at this sort of thing by now, let's try counting something that's a bit more difficult - the number of two-regular simple graphs of a fixed size. Recall that a simple graph is 2-regular if and only if it satisfies the schema:
\[
    \twor: (\forall x)(\exists^{=2}y)Lxy
\]
which is equivalent to 
\[
    (\forall x)(\exists y)(\exists z)(y\neq z\wedge(\forall w)(Lxw\equiv(w=y\vee w=z)))
\]

Let $S$ be the conjunction of $\twor$ and $\sg$. Let's calculate $\card{\modn{S}{6}}$, ie the number of 2-regular simple graphs of size 6. 

\begin{example}
Let $S$ be the conjunction of $\twor$ and $\sg$. What is $\card{\modn{S}{6}}$, ie how many 2-regular simple graphs of size 6 are there? 
\end{example}

As discussed earlier, if $A$ is finite (as it is in the question above) and $A \models S$, then $A$ is a disjoint union of cycles. It follows that if $A \in \modn{S}{6}$ then $A$ is either a disjoint union of two triangles, or a single hexagon (since a triangle is the minimal 2-regular cycle, these are the only possibilities). So in order to complete our calculation, we just need to determine how many distinct ways we can label a structure of one or the other of these shapes. How can we do that?

Let's consider the two triangles case first. Suppose the unlabeled structure $\mathbb{T}$ consists of two triangles, call them the top triangle and the bottom triangle. We can label the top triangle with any set $X\subseteq[6]$ of size three, leaving $[6]-X$ to label the bottom triangle. At first blush, this suggests that there are $\binom{6}{3}$ distinct labelings of $\mathbb{T}$. But notice that we get the same labeled structure, if we use $[6]-X$ to label the top triangle, and $X$ to label the bottom triangle. It follows that there are $\binom{6}{3}/2=10$ distinct labelings of $\mathbb{T}$. 

Next, suppose the unlabeled structure $\mathbb{H}$ consists of a single hexagon. We can use our prior calculation that there are $6!$ strict linear orders of [n] to calculate the number of distinct labelings of $\mathbb{H}$. The rotational symmetry of the hexagon means that if we ``wrap the ordering around'' a hexagon from some fixed starting point, each of the $6$ possible rotations of the labelled hexagon (each corresponding to a different linear order) are equivalent. Moreover it is clear that the reverse of any order gives the same labeling as the order itself. Thus, the total number of labelings of $\mathbb{H}$ is $6!/(6\cdot2)=60$. It follows that $\card{\modn{S}{6}}=10+60=70$.