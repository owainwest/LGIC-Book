\subsection*{Functions, Tournaments, and Orderings}
You may already have encountered functions, such as the mapping $f$ that sends a real number $x$ to its square $x^2$. You were probably given the following definition of a function:

\begin{definition}
A function is any mapping $f$ from a \emph{domain} $A$ to a \emph{codomain} $B$ (written $f: A \rightarrow B$) such that no $a \in A$ maps to more than one element in $b$.\footnote{Here, we only consider \emph{unary} functions - eg, functions of one argument. The definition could be amended for general $n$-ary functions by letting $\overline{a} \in A^n$ denote an $n$-tuple of elements in the domain.}
\end{definition}

Herein (unless otherwise noted) we will always consider functions whose domain and codomain are the same; this will allow us to consider functions as special types of binary relations on some universe of discourse. 

You have probably seen the function $f(x) = x^2$ represented in cartesian coordinates via a graph, that is, the set of all ordered pairs of real numbers $\op{x}{x^2}$ for $x\in\R$. This suggests that a function can be thought of as a specific type of binary relation. In the example $f(x) = x^2$, this means that we consider the Cartesian graph as a structure, eg a directed graph $A$ with $U^A=\R$ and $L^A=\{\op{x}{x^2} \mid x\in\R\}$. This structure satisfies the following schemata.
\begin{itemize}
\item
\tot: $(\forall x)(\exists y)Lxy$
\item
\sv: $(\forall x)(\forall y)(\forall z)((Lxy\wedge Lxz)\supset y=z)$
\end{itemize}

The first of these says that the $L$ is \emph{total}, that is, everything is related (here think ``mapped to'') at least one thing, and the second says that $L$ is \emph{single-valued}, that is, everything is mapped to at most one thing. 

Of course, \sv\ serves as an alternative definition of a function. The conjunction of \sv\ and \tot\, which we abbreviate to \fun, says that $L$ is a total function, that is, if $A\models\fun$, then $L^A$ is the graph of a total function with domain $U^A$ and range (contained in) $U^A$. 


There are special types of functions which will be interesting to us: namely, \emph{injections}, \emph{surjections}, and \emph{bijections}. 

\begin{definition}
An \emph{injection} (also called a 1-1 function) is a function which maps distinct elements of the domain to distinct elements of the codomain. In other words, no two different elements of the domain map to the same element in the codomain. We schematize this as
\[
    \inj: (\forall x)(\forall y)(\forall z)((Lxz\wedge Lyz)\supset x=y)
\]
\end{definition}


You may be familiar with the idea in terms of the ``horizontal line rule'', which says that if a horizontal line crosses the (Cartesian) graph of a function in more than one point, that function is not injective.

\begin{aside}
    Use the horizontal line rule to show that $f(x) := x^2$ is not injective. Give a general relation between distinct elements $a, b$ which witness that $f$ is not injective. 
\end{aside}


\begin{definition}
A \emph{surjection} (also called an onto function) is a function in which every element of the codomain is mapped to by some element of the domain. In other words, no element of the codomain is not mapped to. Schematically:
\[
    \sur: (\forall x)(\exists y)Lyx
\]
\end{definition}

\begin{aside}
    Show that $f(x) := x^2$ is not surjective when our universe of discourse is $\mathbb{R}$, the set of real numbers. Fixing the domain as $\mathbb{R}$, give a codomain which would make the function surjective. 
\end{aside}

\begin{definition}
A \emph{bijection} is a function which is both injective and surjective. Schematically
\[
    \bij: \inj \land \sur
\]
\end{definition}

\begin{aside}
    Show that $f(x) := x^3$ is a bijection. 
\end{aside}

We have now seen examples of functions which are either bijective (the cubing function) and functions which are neither injective nor surjective (the squaring function). Is it possible to have a function which is injective but not surjective, or surjective but not injective?

For any structure $A$ with finite universe of discourse, $A \models \fun \land \inj$ iff $A \models \fun \wedge \sur$, ie any injective function on a finite structure is surjective, and vice versa. 

\begin{aside}
    Prove this. 
\end{aside}

The same does not hold for structures with infinite universes - in fact, the influential logician Dedekind used this distinction as his definition of infinitude. For example, consider the structure $A$ where $U^A=\N$ and $L^A=\{\op{n}{n+1}\mid n\in\N\}$ and observe that $A\models\fun\wedge\inj\wedge\neg\sur$. It is similarly easy to construct functions which are surjections but not injections, for example, the function on \N\ that maps a number $n$ to $\lceil n/2\rceil$.  A set $X$ is said to be \emph{Dedekind infinite} if and only if there is a function with domain $X$ and range contained in $X$ which is injective but not surjective.


We now touch briefly on the topic of multivariate functions; we restrict our attention to binary functions whose graphs we represent as the interpretation of a triadic predicate symbol $R$. The following schema \bfun\ expresses both totality and single-valuedness, that is, a structure $A$ satisfies \bfun\ if and only if $R^A$ is the graph of a total binary function on $U^A$..
\begin{itemize}
\item
\bfun: $(\forall x)(\forall y)(\exists
z)(\forall 
w)(Rxyw\equiv w = z)$
\end{itemize}
The next schema \binj\ schematizes the notion of injection for binary functions, that is, a structure $A$ satisfies the conjunction of \bfun\ and \binj\ if and only if $R^A$ is the graph of an injective binary function.
\begin{itemize}
\item 
\binj: $(\forall v)(\forall w)(\forall x)(\forall y)(\forall z)((Rvwz \wedge Rxyz)
\supset (v = x \wedge w = y))$
\end{itemize}
If $A$ is a finite structure and $A\models\bfun\wedge\binj$, then $\card{U^A}=1$.

\begin{aside}
    Prove this.
\end{aside}

On the other hand, we noted that the binary function which maps a pair of positive integers $m$ and $n$ to $2^m\cdot3^n$ is an injection (this follows from the fundamental theorem of arithmetic). This shows that there are at least as many positive integers as there are positive rational numbers, since every positive rational number can be represented by a pair of integers. This may seem odd, since, in their usual order, between any two positive integers there are infinitely many rational numbers. 


\subsection*{Tournaments and Orderings}


\begin{definition}
We say that a directed graph is \emph{asymmetric} if no pair of its nodes have ``edges in both directions'', ie
\[
  \asy: (\forall x)(\forall y)(Lxy\supset \neg Lyx)  
\]
\end{definition}

\begin{definition}
We say that a directed graph is \emph{comparable} if every pair of distinct nodes has at least one edge between them, eg
\[
    \comp: (\forall x)(\forall y)((x\neq y\supset (Lxy\vee Lyx)
\]
\end{definition}

\begin{definition}
We say a directed graph is a \emph{tournament} iff it is both asymmetric and comparable, eg
\[
    \tour: \asy \land \comp
\]
\end{definition}
The intuitive justification for the name ``tournament'' is that round-robin tournaments involve each team playing every other team once, and that each game between teams $a, b$ induces an ``edge'' which points from the victor to the loser. 

Let's pick out a particularly important class of tournaments, those without cycles. We will these as the \emph{transitive} tournaments. 

\begin{definition}
We say that a tournament is \emph{transitive} iff the edge relation is transitive, eg
\[
    \trans: (\forall x)(\forall y)(\forall z)(Lxy\supset(Lyz\supset Lxz))
\]
\end{definition}

\begin{definition}
A \emph{strict linear order} is a transitive tournament, eg
\[
    \slo: \trans \land \tour
\]
\end{definition}