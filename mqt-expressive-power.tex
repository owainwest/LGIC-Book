\subsection{Expressive Power}

\subsubsection*{The expressive power of monadic quantification theory}

With these results in hand, we proceed to analyze the expressive power of monadic schemata. For simplicity's sake, we'll continue to focus on the vocabulary consisting of the monadic predicate letters $F$ and $G$. First, some definitions. 
\begin{itemize}
\item 
A list of pure monadic schemata is \textit{succinct} if and only if no two schemata on the list are equivalent. 
\item 
A pure monadic schema \textit{implies a list of schemata} if and only if it implies every schema on the list.
\item The \textit{power} of a pure monadic schema is the length of a longest succinct list of pure monadic schemata it implies.  
\end{itemize}

Now, the main question: how expressive is MQT?

\begin{question}\label{succinct-q}
What is the length of a longest succinct list of pure monadic schemata (in the vocabulary consisting of just the monadic predicate letters $F$ and $G$)? In other words, how many propositions can MQT express?
\end{question}
\emph{Answer}:
It follows immediately from Corollary \ref{monad-cor}, part (\ref{equiv-item}) that the length of a longest such list is $2^{15}$, since a schema is determined, up to equivalence, by which of the structures $A_1,\ldots,A_{15}$ satisfy it. 
\begin{question}\label{pow-q}
For which numbers $n$ is there a schema $S$ whose power is $n$?
\end{question}
\emph{Answer}:
It follows from Corollary \ref{monad-cor}, parts (\ref{imp-item}) and (\ref{equiv-item}), that the power of a schema $S$ is determined by the size $j$ of $\{i\mid A_i\models S\ \mbox{and}\ 1\leq i\leq 15\}$, in particular, the power of $S$ is $2^{15-j}$; for pure schemata $S$, $j$ may be any number between 0 and 15. This answers Question \ref{pow-q}.

\begin{definition}
If $X$ is a finite set, we write $\card{X}$ for the number of members of $X$.

If $S$ is a schema, we write $\modn{S}{n}$ for the set of structures $A$ such that $A\models S$ and $U^A=\{1,\ldots, n\}.$
\end{definition}
\begin{question}\label{mspec-q}
What is the length of a longest succinct list of pure schemata $S$ such that $\modn{S}{4}=4$?
\end{question}

\emph{Answer}:
Let $\mathbb{V} = \{A\mid U^A=\{1,2,3,4\}\}$. Recall that $A\approx_M B$ if and only if for all pure monadic schemata $S$, $A\models S$ if and only $B\models S$. For $A\in \mathbb{V}$, let $\bm\hat{A}=\{B\in \mathbb{V}\mid B\approx_M A\}$.

\begin{aside}
    $\bm\hat{A}$ is the monadic similarity \emph{equivalence class} of $A$, ie all structures which are monadically similar to $A$. Generally, an equivalence class of an object $N$ is the set of all objects which are equivalent to $N$ under some equivalence relation. 
\end{aside}

In order to answer the question, it suffices to determine the size of $\bm\hat{A}$ for each $A\in\mathbb{V}$. First, note that the size of $\bm\hat{A}$ is determined by the number of types realized by $A$. We computed these sizes:
\begin{itemize}
\item
If $A$ realizes exactly 1 type, then the size of $\bm\hat{A}$ is 1, since monadically similar structures realize the same types, and there is only 1 way to place 4 elements into one given type. There are $\binom{4}{1}$ structures in $\mathbb{V}$ satisfying exactly 1 type (since there are 4 choices of which type is realized).
\item
If $A$ realizes exactly 2 types, then the size of $\bm\hat{A}$ is $2^4-2$ (since there are $2^4 - 2$ ways of distrubuting 4 elements into two given types such that each type is nonempty). There are $\binom{4}{2}$ structures in $\mathbb{V}$ satisfying exactly 2 types (since there are $\binom{4}{2}$ choices for which two types are realized).
\item
If $A$ realizes exactly 3 types, then the size of $\bm\hat{A}$ is $\binom{4}{2}\cdot3!$ (since there are $\binom{4}{2}\cdot3!$ ways of distrubuting 4 elements into three given types such that each type is nonempty). There are $\binom{4}{3}$ structures in $\mathbb{V}$ satisfying exactly 3 types.
\item
If $A$ realizes exactly 4 types, then the size of $\bm\hat{A}$ is 4! (since there are $4!$ ways of ordering the 4 elements, with the $i^{th}$ element in our order being places in $T_i$). There are $\binom{4}{4} = 1$ structures in $\mathbb{V}$ satisfying exactly 4 types.
\end{itemize}
\begin{aside}
    If the size of $\bm\hat{A}$ is confusing for any of the above, try to count these yourself! If you're still stuck, come into Office Hours and we'll be happy to help. 
\end{aside}

By Theorem \ref{types-lem}, if $A \models S$ then for all $A_i \in \bm\hat{A}$, $A_i \models S$. It follows that the answer to Question \ref{mspec-q} is 1; in particular, one such list consists of the single schema
\[(\forall x)(Fx\wedge Gx)\vee(\forall x)(Fx\wedge \neg Gx)\vee(\forall x)(\neg Fx\wedge Gx)\vee(\forall x)(\neg Fx\wedge \neg Gx).\]

