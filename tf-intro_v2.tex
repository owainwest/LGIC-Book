\subsection{Introduction to Truth-Functional Logic}
Throughout the course we will see a few different systems for formalizing statements. Each consists of a formal language to represent statements, and a way to interpret the meaning of statements in that language. Truth-functional logic is the simplest of these systems we will learn. 

\subsubsection*{Components of Truth Functional Logic}
\begin{enumerate}
    \item Language (the \emph{Syntax})
    \begin{enumerate}
        \item sentence letters
        \item connectives
    \end{enumerate}
    \item Interpretation (the \emph{Semantics})
    \begin{enumerate}
        \item A function that assigns $\top$ or $\bot$ (true or false) to each sentence letter, called a \textbf{truth-assignment}
        \item Fixed \textbf{truth-functional semantics} for each connective
    \end{enumerate}
\end{enumerate}

\textbf{Sentence letters} such as $p, q, r, \ldots$ schematize statements (in natural language) which are true or false, and \textbf{connectives} such as $\wedge, \vee, \neg, \supset, \ldots$ are used to combine sentence letters into compound schemata. 

\begin{aside}
Statements are sentences whose truth or falsity is independent of context of utterance. For example, the sentence ``I am bald'' is not a statement, since the truth or falsity of a given utterance of this sentence depends not only on the speaker and the time of utterance, but also on whatever subtle contextual factors might partially restrict the the range of application of the vague term ``bald.'' On the other hand, ``eight times seven is fifty-four'' is as a statement, since it's truth or falsity (in this case falsity) is independent of context of utterance. Neither of the sentences ``is eight times seven fifty-four?'' nor ``please, let eight times seven be fifty-four,'' is a statement. Truth-functional logic deals with the truth or falsity of \emph{statements} only, and we use sentence letters exclusively to schematize statements. 
\end{aside}