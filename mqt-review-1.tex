\subsection{Review}
\begin{mdframed}[linewidth=1]
\section*{Concept Review}
\textbf{Definitions}
\begin{itemize}
    \item A \emph{one variable open schema} is a truth-functional compound of atomic schemata with the same variable $x$ (for example, $(Fx \land Gx)$). 
    \item A \emph{simple monadic schema} is a schema in which some quantifier binds a one-variable open schema (for example, $(\forall x)(Fx \land Gx)$). 
    \item A \emph{pure monadic schema} is a truth-functional compound of simple monadic schemata (for example, $(\forall x)(Fx \land Gx) \vee (\exists x)(Gx)$).
    \item A schema $S$ is satisfiable iff there is at least one structure $A$ such that $A \models S$. Note that this is analogous to the corresponding definition for TF-logic, the only difference being that $A$ is now a structure, not a truth-assignment. 
    \item A schema $S$ is valid iff for all structures $A$, $A \models S$
    \item A schema $S$ implies a schema $T$ iff for all $A$, if $A \models S$, then $A \models T$
    \item Schemata $S$ and $T$ are equivalent iff they are satisfied by exactly the same structures. 
    \item A structure $A$ is said to ``be a counterexample to the claim that a schema $S$ implies a schema $T$'' iff $A \models S$ and $A \not \models T$. 
    \item A structure $A$ is said to ``witness the inequivalence of schemata $S$ and $T$'' iff ($A \models S$ and $A \not \models T$), or ($A \models T$ and $A \not \models S$). 
    \item Structures $A, B$ are said to be \emph{monadically similar} ($A \approx_M B$) iff they satisfy the same pure monadic schemata. 
    \item A function $h$ is \emph{surjective} (or \emph{onto})  if everything in the codomain is mapped to by $h$ (equivalently, the image of the function - which is the set of all things that get mapped to - is equal to the codomain). In the language of first order logic, the criterion for surjectivity is
        \[
             (\forall b \in B)(\exists a \in A)(h(a) = b)
         \] 
    \item A \emph{homomorphism $h$ from $A$ onto $B$} satisfies the following three properties:
    \begin{itemize}
        \item $h$ is a function from $A$ to $B$.
        \item $h$ is \emph{surjective (onto)}. 
        \item $h$ is ``structure preserving''. This just means that the extensions of predicate letters are preserved under the homomorphism. You can think of this as preservation of the types of elements in the ``Type View,'' or preservation of sections in the Venn diagram in the ``Venn View.'' In the language of first order logic, the criterion for this is
        \[
            \text{for all  predicate letters }P,\ (\forall a \in A)(a \in P^A \equiv h(a)\in P^B)
        \]
    \end{itemize}

\end{itemize}

\textbf{Quantifiers}
$\forall x$ is read ``for all $x$'', ``for every $x$'', or ``for each $x$''. A sentence of the form $$(\forall x)(\text{some statement about }x)$$ is true just in case ``some statement about $x$'' is true no matter what $x$ is. 

$\exists x$ is read ``there exists an $x$''. A sentence of the form $$(\exists x)(\text{some statement about }x)$$ is true just in case ``some statement about $x$'' is true of at least one thing $x$. 

\textbf{(Unary) Predicates}
\emph{Predicates}  are true of some things and not true of others. They correspond to the ``$\text{some statement about }\bigcirc$'' segment of the aforementioned sentences. For example, ``$\bigcirc$ is an even number'' is a predicate that is true just of the elements of the set of even integers, and false of all other things. The set of things of which a predicate is true is called its \emph{extension}. For example, the extension of ``$\bigcirc$ is an even number'' is the set $\{0, 2, -2, 4, -4...\}$. 

We schematize predicates by \emph{predicate letters}. For example, we might say that $Fx$ represents the statement that ``$x$ is an even number''.

\textbf{Structures}
In truth-functional logic, the truth value of a schema was relative to truth-assignments to its sentence letters. Structures play the same role for monadic quantification theory as truth-assignments do for truth-functional logic. 
 
A structure $A$ consists of:
\begin{itemize}
    \item A set called the \emph{universe of A}, written $U^A$. This set is the range for our variables of quantification. 
    \begin{itemize}
        \item For example, if I asserted that $(\forall x)(x \text{ is even} \vee x \text{ is odd})$, you'd probably assume that I was talking about all integers, not all things in general. In this case, the universe would (implicitly) be the set of integers. In ordinary speech, this implicit universe is contextually determined. When specifying a structure to interpret some monadic quantificational schemata, we explicitly identify the universe of discourse.
            \end{itemize}
    \item The extensions of some monadic predicate letters ($F^A, G^A$, emph{etc}.), all of which are (arbitrary) subsets of the universe. 
    \begin{itemize}
        \item Let $F^A$ be the set of even numbers, and $G^A$ be the set  of odd numbers. Then, within the universe of integers, the statement $(\forall x)(Fx \vee Gx)$ is a true statement, asserting that $(\forall x)(x \text{ is even} \vee x \text{ is odd})$.
    \end{itemize}
\end{itemize}
If a sentence is true relative to some structure, we write $A \models S$ and say that ``$A$ satisfies $S$'' (as we did for TF-logic), ``$A$ is a model of $S$'', or ``$S$ is true in $A$''.

\textbf{Bound vs. Free Variables}
A variable $x$ is said to be \emph{bound} if it is within the scope of some quantifier. $x$ is said to be \emph{free} if it is not bound. The truth of sentences including free variables cannot be evaluated without an \emph{assignment} of a value to those free variables. For example, the truth value of the sentence ``$x$ is an even number'' cannot be determined without some assignment of a value (say, 2 or 3) to $x$. However, the sentence $(\forall x)(x \text{ is an integer} \supset x \text{ is even})$ can be evaluated (and is, of course, false). 

\textbf{3 Views of Structures}
We have three equivalent ways of looking at structures, which are
\begin{enumerate}
    \item \textbf{The Canonical View}, which involves specifying the universe of discourse and extensions of predicates as sets. 
    \item \textbf{The Types View}, which involves drawing a table with sections for each ``type''.
    \item \textbf{The Venn View}, which involves drawing a venn diagram, wherein the circles represent extensions of predicates. 
\end{enumerate}


\textbf{Realizing Types} 
We say that a structure realizes a type $T_i$ iff there is some element of the structure in the quadrant $T_i$ in the types view of our structure. 

For example, with predicate letters $F, G$, the types would then be 
\begin{itemize}
    \item $T_1(x): Fx \land Gx$

    \item $T_2(x): Fx \land \lnot Gx$

    \item $T_3(x): \lnot Fx \land Gx$

    \item $T_4(x): \lnot Fx \land \lnot Gx$
\end{itemize}
and then we would say that a structure realizes type $T_i$ iff it makes the sentence $(\exists x)(T_i(x))$ true. 

\textbf{The Small Model Theorem}
The Small Model Theorem states:
\begin{theorem}
Let $S$ be a pure monadic schema over $n$ predicate letters. If $S$ is satisfiable, then there is a structure $A$ with $|U^A| \leq 2^n$ with $A \models S$. 
\end{theorem}

Corollaries to the Small Model Theorem include the algorithmic decidability of the satisfiability problem for schemata of MQT, and the fact that there are only finitely many schemata up to equivalence whose predicate letters are drawn from a fixed finite set.
\iffalse
This immediately gives as a corollary the algorithmic decidability of satisfiability for MQT. To see this, we noted the following equivalence (the \emph{contrapositive})
\[
    (p \supset q) \equiv (\lnot q \supset \lnot p)
\]
Then the contrapositive of our theorem is: \emph{if there is no structure $A$ with $|A| \leq 2^n$ such that $A \models S$, then $A$ is not satisfiable}. Thus, we have to check only a finite number of models to see if $S$ is satisfiable or not, and hence satisfiability is algorithmically decidable. 
\fi
\end{mdframed}



\newpage
\begin{mdframed}[linewidth=1]
\section*{Problems}
For the later problems, it will be immensely useful to you to draw tables that look like this. Try to interpret the schema as statements about which quadrants of the table can/must have elements in them for the schema to be satisfied or falsified (this is the ``types view''). 

\begin{center}
\begin{tabular}{l|l|l}
 & $Fx$ & $\lnot Fx$ \\ \hline
$Gx$  &  &  \\ \hline
$\lnot Gx$ &  & 
\end{tabular}
\end{center}

\begin{enumerate}
    \item Are the following sentences equivalent? If not, does one imply the other?
    \[
        S: (\forall x)(Px)
    \]
    \[
        T: \lnot(\exists x)(\lnot Px)
    \]

    \item Are the following sentences equivalent? If not, does one imply the other?
    \[
        S: (\exists x)(Px)
    \]
    \[
        T: \lnot(\forall x)(\lnot Px)
    \]

    \item Are the following sentences equivalent? If not, does one imply the other?
    \[
        S: (\forall x)(Px) \land (\forall x)(Qx)
    \]
    \[
        T: (\forall x)(Px \land Qx)
    \]

    \item Are the following sentences equivalent? If not, does one imply the other?
    \[
        S: (\exists x) (Px) \land (\exists x)(Qx)
    \]
    \[
        T: (\exists x)(Px \land Qx)
    \]

    \item Are the following sentences equivalent? If not, does one imply the other?
    \[
        S: (\forall x)(Px) \vee (\forall x)(Qx)
    \]
    \[
        T: (\forall x)(Px \vee Qx)
    \]

    \item     Let
    \[
        S : (\exists x)(Fx \land Gx) \land (\exists x)(\lnot Fx \land Gx) \land (\exists x)(Fx \land \lnot Gx) \land (\exists x)(\lnot Fx \land \lnot Gx)
    \]
    \[
        T : (\forall x)(Fx \equiv Gx)
    \]
    \begin{enumerate}
        \item How many structures with universe $U = \{1, 2, 3\}$ are counterexamples to the claim that $S$ implies $T$?

        \item How many structures with universe $U = \{1, 2, 3, 4, 5\}$ are counterexamples to the claim that $S$ implies $T$?
    \end{enumerate}

    \item Let 
    \[
        S: (\forall x)(Fx \oplus Gx)
    \]
    \[
        T: (\forall x)(Fx \equiv Gx)
    \]
    How many structures are there with universe $U = \{1, 2, 3, 4, 5\}$ which witness the inequivalence of $S$ and $T$?

    \item Let 
    \[
        S: (\exists x)(Fx \land Gx)
    \]
    \[
        T: (\forall x)(Fx \vee Gx)
    \]
    How many structures are there with universe $U = \{1, 2, 3, 4, 5\}$ which witness the inequivalence of $S$ and $T$?

    \item  Let 
\[
    S: (\forall x)(Fx \oplus Gx)
\]
\[
    T: (\forall x)(Fx) \oplus (\forall x)(Gx)
\]
Given universe $U = \{1, 2, 3, 4, 5\}$, how many counterexamples are there to the claim that $S$ implies $T$?

\item Let 
\[
    S: (\forall x)(Fx \supset Gx)
\]
\[
    T: (\forall x)(Fx) \supset (\forall x)(Gx)
\]
Given universe $U = \{1, 2, 3, 4, 5\}$, how many counterexamples are there to the claim that $S$ implies $T$?
\end{enumerate}
\end{mdframed}

\newpage
\begin{mdframed}[linewidth=1]
\section*{Solutions}
\begin{enumerate}
    \item These are equivalent. The former asserts that everything has property $P$. The latter asserts that nothing lacks property $P$. These are equivalent statements. 

    \item These are equivalent. The former asserts that there is something with property $P$. The latter asserts that it is not the case that everything lacks $P$. These are equivalent statements. 

    \item These are equivalent. The former asserts that everything is $P$, as well as that everything is $Q$. The latter asserts that everything is both $P$ and $Q$.

    \item These are not equivalent, but $T$ implies $S$. $S$ asserts that there is something that is $P$, and there is something that is $Q$. $T$ asserts that there is something that is both $P$ and $Q$. Clearly if $T$ is true, $S$ must be as well. Hence $T$ implies $S$. The inequivalence of $S$ and $T$ is witnessed by the structure $A$ with $U^A = \{1, 2\}$, $P^A = \{1\}$, $Q^A = \{2\}$. Then $A \models S$ but $A \not \models T$. 

    \item These are not equivalent, but $S$ implies $T$. $S$ asserts that everything is $P$ or everything is $Q$. $T$ asserts that everything is at least one of $P$ or $Q$. Clearly if $S$ is true then so is $T$, hence $S$ implies $T$. The inequivalence of $S$ and $T$ is witnessed by the structure $A$ with $U^A = \{1, 2\}$, $P^A = \{1\}$, $Q^A = \{2\}$. Then $A \models T$ but $A \not \models S$. 

    \item 
    \begin{enumerate}
        \item 0. To satisfy $S$, there must be at least one element which is each of $(Fx \land Gx)$, $(Fx \land \lnot Gx)$, $(\lnot Fx \land Gx)$, and $(\lnot Fx \land \lnot Gx)$. We only have three elements though, so at least one of those categories won't have an element in it. Hence $S$ can't be satisfied in this universe. Hence there are no structures $A$ such that $A \models S$ and $A \not \models T$. 

        \item $\binom{5}{2}\cdot 4! = 240$. To satisfy $S$, there must be at least one element which is each of $(Fx \land Gx)$, $(Fx \land \lnot Gx)$, $(\lnot Fx \land Gx)$, and $(\lnot Fx \land \lnot Gx)$. Think of these categories as ``boxes'' into which we are placing elements of our universe (draw a table to help yourself think through problems like this!). There are $\binom{5}{2}\cdot 4!$ ways to satisfy $S$. The $\binom{5}{2}$ term results from choosing two items from $U$ which will go into the same ``box''. Henceforth we think of these two elements as now being a ``package deal''. The $4!$ term is the number of ways we can order our four things (the three single elements, plus our ``package deal'') into the four ``boxes''. Note that any structure that satisfies $S$ cannot satisfy $T$. Hence all of the structures satisfying $S$ are counterexamples to the claim that $S$ implies $T$, and we have our answer. 
    \end{enumerate}

    \item $2^6$. Note that if a structure satisfies $S$, it cannot satisfy $T$, and similarly if one satisfies $T$, it cannot satisfy $S$. Hence it suffices to count the number of ways $S, T$ can each be satisfied and add those results together.

    To satisfy $S$, every element of $U$ must be either $(Fx \land \lnot Gx)$ or $(\lnot Fx \land Gx)$. This corresponds to two choices for each of our $5$ elements, meaning there are $2^5$ ways to satisfy $S$. Note that none of these structures satisfy $T$.

    Similar reasoning suffices to show that there are $2^5$ ways to satisfy $T$. None of these structures satisfy $S$.

    Our answer is then $2^5 + 2^5 = 2^6$. 

    \item $4^5 - 2\cdot 3^5 + 2^6 = 602$. 

    Let's start by counting the number of structures which are counterexamples to the claim that $T$ implies $S$, since that direction is easier. For $T$ to be true, every element must satisfy one of the types $(Fx \land Gx)$, $(Fx \land \lnot Gx)$, or $(\lnot Fx \land Gx)$. For $S$ to be false, no element can satisfy the type $(Fx \land Gx)$. Hence, for $T$ to be satisfied and $S$ to be falsified, every element must satisfy either $(Fx \land \lnot Gx)$ or $(\lnot Fx \land Gx)$. Since we have two choices per element, there are $2^5$ total ways to satisfy $T$ and falsify $S$. 

    How many ways are there to satisfy $S$ and falsify $T$? Let's begin by counting the ones that satisfy $S$. These are the structures which have at least one element of type $(Fx \land Gx)$. There are $4^5$ structures in total, $3^5$ of which have no element of type $(Fx \land Gx)$. Hence, there are $4^5 -3^5$ structures satisfying $S$. From this number, we need to subtract the number that also satisfy $T$. $3^5$ structures satisfy $T$ (the structures in which every element is of one of the types $(Fx \land Gx)$, $(Fx \land \lnot Gx)$, or $(\lnot Fx \land Gx)$), so we subtract that from the number which satisfy $S$ to get $4^5 - 2\cdot 3^5$. Notice, however, that we \emph{took away all structures whose elements were exclusively of type $(Fx \land \lnot Gx)$ or $(\lnot Fx \land Gx)$ twice - once when we were counting the number that satisfied $S$, and another time when counting the number that satisfed $T$}. Hence we ``double subtracted'' $2^5$ possibilities, and we must add this back. This gives the result that there are $4^5 - 2\cdot 3^5 + 2^5$ structures satisfying $S$ which do not satisfy $T$. 

    Adding these two results together, we get $2^5 + (4^5 - 2\cdot 3^5 + 2^5) = 4^5 - 2\cdot 3^5 + 2^6 = 602$. 

    \item $2^5 - 2$. Note that $S$ is satisfied by $2^5$ structures (each element can be in either the box $(Fx \land \lnot Gx)$ or the box $(Gx \land \lnot Fx)$). Of these, only two satisfy $T$ (the one in which all elements are in $(Gx \land \lnot Fx)$, and the one in which all elements are in $(Fx \land \lnot Gx)$). Hence, there are $2^5 - 2$ structures which satisfy $S$ and falsify $T$. 

    \item $0$. There are $3^5$ structures satisfying $S$. Of these, none falsify $T$. 
\end{enumerate}
\end{mdframed}