Though the course has no specific mathematical prerequisites, a general familiarity with the set of integers and some of its basic properties will be assumed. We collect here some useful facts and notations that will appear from time to time throughout the course. We'll add more as the need arises.
\begin{enumerate}
\item 
Notations for important sets of numbers
\begin{itemize}
\item 
$\int=\{\ldots -2,-1,0,1,2,\ldots\}$ (the integers)
\item 
$\nnint=\{0,1,2,\ldots\}$ (the non-negative integers, a.k.a. the natural numbers)
\item 
$\pint=\{1,2,3,\ldots\}$ (the positive integers)
\end{itemize}
\item 
Important facts about numbers
\begin{itemize}
\item 
The Least Number Principle: If $X$ is a nonempty subset of \nnint, then $X$ has a least element.
\item 
Principle of Mathematical Induction: If $X$ is a subset of \nnint, and $0\in X$, and for every $i$, if $i\in X$, then $i+1\in X$, then $X=\nnint$.
\item The Pigeonhole Principle: If you distribute $m$ pigeons into $n$ pigeonholes and $m>n$, then some hole contains more than one pigeon.
\end{itemize}
\item Unique Factorization into Primes: Recall that $p\in\pint$ is \emph{prime} if and only if $p\neq1$ and $p$ is divisible only by $1$ and $p$. Every $n\in\pint$ with $n\neq 1$ can be written uniquely as $p_1^{a_1}\cdots p_n^{a_n}$ where each $p_i$ is prime and each $a_i\geq 1$. 
\end{enumerate}